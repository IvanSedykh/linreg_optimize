\documentclass[11pt]{article}

    \usepackage[breakable]{tcolorbox}
    \usepackage{parskip} % Stop auto-indenting (to mimic markdown behaviour)
    
    \usepackage{iftex}
    \ifPDFTeX
    	\usepackage[T1]{fontenc}
    	\usepackage{mathpazo}
    \else
    	\usepackage{fontspec}
    \fi

    % Basic figure setup, for now with no caption control since it's done
    % automatically by Pandoc (which extracts ![](path) syntax from Markdown).
    \usepackage{graphicx}
    % Maintain compatibility with old templates. Remove in nbconvert 6.0
    \let\Oldincludegraphics\includegraphics
    % Ensure that by default, figures have no caption (until we provide a
    % proper Figure object with a Caption API and a way to capture that
    % in the conversion process - todo).
    \usepackage{caption}
    \DeclareCaptionFormat{nocaption}{}
    \captionsetup{format=nocaption,aboveskip=0pt,belowskip=0pt}

    \usepackage[Export]{adjustbox} % Used to constrain images to a maximum size
    \adjustboxset{max size={0.9\linewidth}{0.9\paperheight}}
    \usepackage{float}
    \floatplacement{figure}{H} % forces figures to be placed at the correct location
    \usepackage{xcolor} % Allow colors to be defined
    \usepackage{enumerate} % Needed for markdown enumerations to work
    \usepackage{geometry} % Used to adjust the document margins
    \usepackage{amsmath} % Equations
    \usepackage{amssymb} % Equations
    \usepackage{textcomp} % defines textquotesingle
    % Hack from http://tex.stackexchange.com/a/47451/13684:
    \AtBeginDocument{%
        \def\PYZsq{\textquotesingle}% Upright quotes in Pygmentized code
    }
    \usepackage{upquote} % Upright quotes for verbatim code
    \usepackage{eurosym} % defines \euro
    \usepackage[mathletters]{ucs} % Extended unicode (utf-8) support
    \usepackage{fancyvrb} % verbatim replacement that allows latex
    \usepackage{grffile} % extends the file name processing of package graphics 
                         % to support a larger range
    \makeatletter % fix for grffile with XeLaTeX
    \def\Gread@@xetex#1{%
      \IfFileExists{"\Gin@base".bb}%
      {\Gread@eps{\Gin@base.bb}}%
      {\Gread@@xetex@aux#1}%
    }
    \makeatother

    % The hyperref package gives us a pdf with properly built
    % internal navigation ('pdf bookmarks' for the table of contents,
    % internal cross-reference links, web links for URLs, etc.)
    \usepackage{hyperref}
    % The default LaTeX title has an obnoxious amount of whitespace. By default,
    % titling removes some of it. It also provides customization options.
    \usepackage{titling}
    \usepackage{longtable} % longtable support required by pandoc >1.10
    \usepackage{booktabs}  % table support for pandoc > 1.12.2
    \usepackage[inline]{enumitem} % IRkernel/repr support (it uses the enumerate* environment)
    \usepackage[normalem]{ulem} % ulem is needed to support strikethroughs (\sout)
                                % normalem makes italics be italics, not underlines
    \usepackage{mathrsfs}
    

    
    % Colors for the hyperref package
    \definecolor{urlcolor}{rgb}{0,.145,.698}
    \definecolor{linkcolor}{rgb}{.71,0.21,0.01}
    \definecolor{citecolor}{rgb}{.12,.54,.11}

    % ANSI colors
    \definecolor{ansi-black}{HTML}{3E424D}
    \definecolor{ansi-black-intense}{HTML}{282C36}
    \definecolor{ansi-red}{HTML}{E75C58}
    \definecolor{ansi-red-intense}{HTML}{B22B31}
    \definecolor{ansi-green}{HTML}{00A250}
    \definecolor{ansi-green-intense}{HTML}{007427}
    \definecolor{ansi-yellow}{HTML}{DDB62B}
    \definecolor{ansi-yellow-intense}{HTML}{B27D12}
    \definecolor{ansi-blue}{HTML}{208FFB}
    \definecolor{ansi-blue-intense}{HTML}{0065CA}
    \definecolor{ansi-magenta}{HTML}{D160C4}
    \definecolor{ansi-magenta-intense}{HTML}{A03196}
    \definecolor{ansi-cyan}{HTML}{60C6C8}
    \definecolor{ansi-cyan-intense}{HTML}{258F8F}
    \definecolor{ansi-white}{HTML}{C5C1B4}
    \definecolor{ansi-white-intense}{HTML}{A1A6B2}
    \definecolor{ansi-default-inverse-fg}{HTML}{FFFFFF}
    \definecolor{ansi-default-inverse-bg}{HTML}{000000}

    % commands and environments needed by pandoc snippets
    % extracted from the output of `pandoc -s`
    \providecommand{\tightlist}{%
      \setlength{\itemsep}{0pt}\setlength{\parskip}{0pt}}
    \DefineVerbatimEnvironment{Highlighting}{Verbatim}{commandchars=\\\{\}}
    % Add ',fontsize=\small' for more characters per line
    \newenvironment{Shaded}{}{}
    \newcommand{\KeywordTok}[1]{\textcolor[rgb]{0.00,0.44,0.13}{\textbf{{#1}}}}
    \newcommand{\DataTypeTok}[1]{\textcolor[rgb]{0.56,0.13,0.00}{{#1}}}
    \newcommand{\DecValTok}[1]{\textcolor[rgb]{0.25,0.63,0.44}{{#1}}}
    \newcommand{\BaseNTok}[1]{\textcolor[rgb]{0.25,0.63,0.44}{{#1}}}
    \newcommand{\FloatTok}[1]{\textcolor[rgb]{0.25,0.63,0.44}{{#1}}}
    \newcommand{\CharTok}[1]{\textcolor[rgb]{0.25,0.44,0.63}{{#1}}}
    \newcommand{\StringTok}[1]{\textcolor[rgb]{0.25,0.44,0.63}{{#1}}}
    \newcommand{\CommentTok}[1]{\textcolor[rgb]{0.38,0.63,0.69}{\textit{{#1}}}}
    \newcommand{\OtherTok}[1]{\textcolor[rgb]{0.00,0.44,0.13}{{#1}}}
    \newcommand{\AlertTok}[1]{\textcolor[rgb]{1.00,0.00,0.00}{\textbf{{#1}}}}
    \newcommand{\FunctionTok}[1]{\textcolor[rgb]{0.02,0.16,0.49}{{#1}}}
    \newcommand{\RegionMarkerTok}[1]{{#1}}
    \newcommand{\ErrorTok}[1]{\textcolor[rgb]{1.00,0.00,0.00}{\textbf{{#1}}}}
    \newcommand{\NormalTok}[1]{{#1}}
    
    % Additional commands for more recent versions of Pandoc
    \newcommand{\ConstantTok}[1]{\textcolor[rgb]{0.53,0.00,0.00}{{#1}}}
    \newcommand{\SpecialCharTok}[1]{\textcolor[rgb]{0.25,0.44,0.63}{{#1}}}
    \newcommand{\VerbatimStringTok}[1]{\textcolor[rgb]{0.25,0.44,0.63}{{#1}}}
    \newcommand{\SpecialStringTok}[1]{\textcolor[rgb]{0.73,0.40,0.53}{{#1}}}
    \newcommand{\ImportTok}[1]{{#1}}
    \newcommand{\DocumentationTok}[1]{\textcolor[rgb]{0.73,0.13,0.13}{\textit{{#1}}}}
    \newcommand{\AnnotationTok}[1]{\textcolor[rgb]{0.38,0.63,0.69}{\textbf{\textit{{#1}}}}}
    \newcommand{\CommentVarTok}[1]{\textcolor[rgb]{0.38,0.63,0.69}{\textbf{\textit{{#1}}}}}
    \newcommand{\VariableTok}[1]{\textcolor[rgb]{0.10,0.09,0.49}{{#1}}}
    \newcommand{\ControlFlowTok}[1]{\textcolor[rgb]{0.00,0.44,0.13}{\textbf{{#1}}}}
    \newcommand{\OperatorTok}[1]{\textcolor[rgb]{0.40,0.40,0.40}{{#1}}}
    \newcommand{\BuiltInTok}[1]{{#1}}
    \newcommand{\ExtensionTok}[1]{{#1}}
    \newcommand{\PreprocessorTok}[1]{\textcolor[rgb]{0.74,0.48,0.00}{{#1}}}
    \newcommand{\AttributeTok}[1]{\textcolor[rgb]{0.49,0.56,0.16}{{#1}}}
    \newcommand{\InformationTok}[1]{\textcolor[rgb]{0.38,0.63,0.69}{\textbf{\textit{{#1}}}}}
    \newcommand{\WarningTok}[1]{\textcolor[rgb]{0.38,0.63,0.69}{\textbf{\textit{{#1}}}}}
    
    
    % Define a nice break command that doesn't care if a line doesn't already
    % exist.
    \def\br{\hspace*{\fill} \\* }
    % Math Jax compatibility definitions
    \def\gt{>}
    \def\lt{<}
    \let\Oldtex\TeX
    \let\Oldlatex\LaTeX
    \renewcommand{\TeX}{\textrm{\Oldtex}}
    \renewcommand{\LaTeX}{\textrm{\Oldlatex}}
    % Document parameters
    % Document title
    \title{homework}
    
    
    
    
    
% Pygments definitions
\makeatletter
\def\PY@reset{\let\PY@it=\relax \let\PY@bf=\relax%
    \let\PY@ul=\relax \let\PY@tc=\relax%
    \let\PY@bc=\relax \let\PY@ff=\relax}
\def\PY@tok#1{\csname PY@tok@#1\endcsname}
\def\PY@toks#1+{\ifx\relax#1\empty\else%
    \PY@tok{#1}\expandafter\PY@toks\fi}
\def\PY@do#1{\PY@bc{\PY@tc{\PY@ul{%
    \PY@it{\PY@bf{\PY@ff{#1}}}}}}}
\def\PY#1#2{\PY@reset\PY@toks#1+\relax+\PY@do{#2}}

\expandafter\def\csname PY@tok@w\endcsname{\def\PY@tc##1{\textcolor[rgb]{0.73,0.73,0.73}{##1}}}
\expandafter\def\csname PY@tok@c\endcsname{\let\PY@it=\textit\def\PY@tc##1{\textcolor[rgb]{0.25,0.50,0.50}{##1}}}
\expandafter\def\csname PY@tok@cp\endcsname{\def\PY@tc##1{\textcolor[rgb]{0.74,0.48,0.00}{##1}}}
\expandafter\def\csname PY@tok@k\endcsname{\let\PY@bf=\textbf\def\PY@tc##1{\textcolor[rgb]{0.00,0.50,0.00}{##1}}}
\expandafter\def\csname PY@tok@kp\endcsname{\def\PY@tc##1{\textcolor[rgb]{0.00,0.50,0.00}{##1}}}
\expandafter\def\csname PY@tok@kt\endcsname{\def\PY@tc##1{\textcolor[rgb]{0.69,0.00,0.25}{##1}}}
\expandafter\def\csname PY@tok@o\endcsname{\def\PY@tc##1{\textcolor[rgb]{0.40,0.40,0.40}{##1}}}
\expandafter\def\csname PY@tok@ow\endcsname{\let\PY@bf=\textbf\def\PY@tc##1{\textcolor[rgb]{0.67,0.13,1.00}{##1}}}
\expandafter\def\csname PY@tok@nb\endcsname{\def\PY@tc##1{\textcolor[rgb]{0.00,0.50,0.00}{##1}}}
\expandafter\def\csname PY@tok@nf\endcsname{\def\PY@tc##1{\textcolor[rgb]{0.00,0.00,1.00}{##1}}}
\expandafter\def\csname PY@tok@nc\endcsname{\let\PY@bf=\textbf\def\PY@tc##1{\textcolor[rgb]{0.00,0.00,1.00}{##1}}}
\expandafter\def\csname PY@tok@nn\endcsname{\let\PY@bf=\textbf\def\PY@tc##1{\textcolor[rgb]{0.00,0.00,1.00}{##1}}}
\expandafter\def\csname PY@tok@ne\endcsname{\let\PY@bf=\textbf\def\PY@tc##1{\textcolor[rgb]{0.82,0.25,0.23}{##1}}}
\expandafter\def\csname PY@tok@nv\endcsname{\def\PY@tc##1{\textcolor[rgb]{0.10,0.09,0.49}{##1}}}
\expandafter\def\csname PY@tok@no\endcsname{\def\PY@tc##1{\textcolor[rgb]{0.53,0.00,0.00}{##1}}}
\expandafter\def\csname PY@tok@nl\endcsname{\def\PY@tc##1{\textcolor[rgb]{0.63,0.63,0.00}{##1}}}
\expandafter\def\csname PY@tok@ni\endcsname{\let\PY@bf=\textbf\def\PY@tc##1{\textcolor[rgb]{0.60,0.60,0.60}{##1}}}
\expandafter\def\csname PY@tok@na\endcsname{\def\PY@tc##1{\textcolor[rgb]{0.49,0.56,0.16}{##1}}}
\expandafter\def\csname PY@tok@nt\endcsname{\let\PY@bf=\textbf\def\PY@tc##1{\textcolor[rgb]{0.00,0.50,0.00}{##1}}}
\expandafter\def\csname PY@tok@nd\endcsname{\def\PY@tc##1{\textcolor[rgb]{0.67,0.13,1.00}{##1}}}
\expandafter\def\csname PY@tok@s\endcsname{\def\PY@tc##1{\textcolor[rgb]{0.73,0.13,0.13}{##1}}}
\expandafter\def\csname PY@tok@sd\endcsname{\let\PY@it=\textit\def\PY@tc##1{\textcolor[rgb]{0.73,0.13,0.13}{##1}}}
\expandafter\def\csname PY@tok@si\endcsname{\let\PY@bf=\textbf\def\PY@tc##1{\textcolor[rgb]{0.73,0.40,0.53}{##1}}}
\expandafter\def\csname PY@tok@se\endcsname{\let\PY@bf=\textbf\def\PY@tc##1{\textcolor[rgb]{0.73,0.40,0.13}{##1}}}
\expandafter\def\csname PY@tok@sr\endcsname{\def\PY@tc##1{\textcolor[rgb]{0.73,0.40,0.53}{##1}}}
\expandafter\def\csname PY@tok@ss\endcsname{\def\PY@tc##1{\textcolor[rgb]{0.10,0.09,0.49}{##1}}}
\expandafter\def\csname PY@tok@sx\endcsname{\def\PY@tc##1{\textcolor[rgb]{0.00,0.50,0.00}{##1}}}
\expandafter\def\csname PY@tok@m\endcsname{\def\PY@tc##1{\textcolor[rgb]{0.40,0.40,0.40}{##1}}}
\expandafter\def\csname PY@tok@gh\endcsname{\let\PY@bf=\textbf\def\PY@tc##1{\textcolor[rgb]{0.00,0.00,0.50}{##1}}}
\expandafter\def\csname PY@tok@gu\endcsname{\let\PY@bf=\textbf\def\PY@tc##1{\textcolor[rgb]{0.50,0.00,0.50}{##1}}}
\expandafter\def\csname PY@tok@gd\endcsname{\def\PY@tc##1{\textcolor[rgb]{0.63,0.00,0.00}{##1}}}
\expandafter\def\csname PY@tok@gi\endcsname{\def\PY@tc##1{\textcolor[rgb]{0.00,0.63,0.00}{##1}}}
\expandafter\def\csname PY@tok@gr\endcsname{\def\PY@tc##1{\textcolor[rgb]{1.00,0.00,0.00}{##1}}}
\expandafter\def\csname PY@tok@ge\endcsname{\let\PY@it=\textit}
\expandafter\def\csname PY@tok@gs\endcsname{\let\PY@bf=\textbf}
\expandafter\def\csname PY@tok@gp\endcsname{\let\PY@bf=\textbf\def\PY@tc##1{\textcolor[rgb]{0.00,0.00,0.50}{##1}}}
\expandafter\def\csname PY@tok@go\endcsname{\def\PY@tc##1{\textcolor[rgb]{0.53,0.53,0.53}{##1}}}
\expandafter\def\csname PY@tok@gt\endcsname{\def\PY@tc##1{\textcolor[rgb]{0.00,0.27,0.87}{##1}}}
\expandafter\def\csname PY@tok@err\endcsname{\def\PY@bc##1{\setlength{\fboxsep}{0pt}\fcolorbox[rgb]{1.00,0.00,0.00}{1,1,1}{\strut ##1}}}
\expandafter\def\csname PY@tok@kc\endcsname{\let\PY@bf=\textbf\def\PY@tc##1{\textcolor[rgb]{0.00,0.50,0.00}{##1}}}
\expandafter\def\csname PY@tok@kd\endcsname{\let\PY@bf=\textbf\def\PY@tc##1{\textcolor[rgb]{0.00,0.50,0.00}{##1}}}
\expandafter\def\csname PY@tok@kn\endcsname{\let\PY@bf=\textbf\def\PY@tc##1{\textcolor[rgb]{0.00,0.50,0.00}{##1}}}
\expandafter\def\csname PY@tok@kr\endcsname{\let\PY@bf=\textbf\def\PY@tc##1{\textcolor[rgb]{0.00,0.50,0.00}{##1}}}
\expandafter\def\csname PY@tok@bp\endcsname{\def\PY@tc##1{\textcolor[rgb]{0.00,0.50,0.00}{##1}}}
\expandafter\def\csname PY@tok@fm\endcsname{\def\PY@tc##1{\textcolor[rgb]{0.00,0.00,1.00}{##1}}}
\expandafter\def\csname PY@tok@vc\endcsname{\def\PY@tc##1{\textcolor[rgb]{0.10,0.09,0.49}{##1}}}
\expandafter\def\csname PY@tok@vg\endcsname{\def\PY@tc##1{\textcolor[rgb]{0.10,0.09,0.49}{##1}}}
\expandafter\def\csname PY@tok@vi\endcsname{\def\PY@tc##1{\textcolor[rgb]{0.10,0.09,0.49}{##1}}}
\expandafter\def\csname PY@tok@vm\endcsname{\def\PY@tc##1{\textcolor[rgb]{0.10,0.09,0.49}{##1}}}
\expandafter\def\csname PY@tok@sa\endcsname{\def\PY@tc##1{\textcolor[rgb]{0.73,0.13,0.13}{##1}}}
\expandafter\def\csname PY@tok@sb\endcsname{\def\PY@tc##1{\textcolor[rgb]{0.73,0.13,0.13}{##1}}}
\expandafter\def\csname PY@tok@sc\endcsname{\def\PY@tc##1{\textcolor[rgb]{0.73,0.13,0.13}{##1}}}
\expandafter\def\csname PY@tok@dl\endcsname{\def\PY@tc##1{\textcolor[rgb]{0.73,0.13,0.13}{##1}}}
\expandafter\def\csname PY@tok@s2\endcsname{\def\PY@tc##1{\textcolor[rgb]{0.73,0.13,0.13}{##1}}}
\expandafter\def\csname PY@tok@sh\endcsname{\def\PY@tc##1{\textcolor[rgb]{0.73,0.13,0.13}{##1}}}
\expandafter\def\csname PY@tok@s1\endcsname{\def\PY@tc##1{\textcolor[rgb]{0.73,0.13,0.13}{##1}}}
\expandafter\def\csname PY@tok@mb\endcsname{\def\PY@tc##1{\textcolor[rgb]{0.40,0.40,0.40}{##1}}}
\expandafter\def\csname PY@tok@mf\endcsname{\def\PY@tc##1{\textcolor[rgb]{0.40,0.40,0.40}{##1}}}
\expandafter\def\csname PY@tok@mh\endcsname{\def\PY@tc##1{\textcolor[rgb]{0.40,0.40,0.40}{##1}}}
\expandafter\def\csname PY@tok@mi\endcsname{\def\PY@tc##1{\textcolor[rgb]{0.40,0.40,0.40}{##1}}}
\expandafter\def\csname PY@tok@il\endcsname{\def\PY@tc##1{\textcolor[rgb]{0.40,0.40,0.40}{##1}}}
\expandafter\def\csname PY@tok@mo\endcsname{\def\PY@tc##1{\textcolor[rgb]{0.40,0.40,0.40}{##1}}}
\expandafter\def\csname PY@tok@ch\endcsname{\let\PY@it=\textit\def\PY@tc##1{\textcolor[rgb]{0.25,0.50,0.50}{##1}}}
\expandafter\def\csname PY@tok@cm\endcsname{\let\PY@it=\textit\def\PY@tc##1{\textcolor[rgb]{0.25,0.50,0.50}{##1}}}
\expandafter\def\csname PY@tok@cpf\endcsname{\let\PY@it=\textit\def\PY@tc##1{\textcolor[rgb]{0.25,0.50,0.50}{##1}}}
\expandafter\def\csname PY@tok@c1\endcsname{\let\PY@it=\textit\def\PY@tc##1{\textcolor[rgb]{0.25,0.50,0.50}{##1}}}
\expandafter\def\csname PY@tok@cs\endcsname{\let\PY@it=\textit\def\PY@tc##1{\textcolor[rgb]{0.25,0.50,0.50}{##1}}}

\def\PYZbs{\char`\\}
\def\PYZus{\char`\_}
\def\PYZob{\char`\{}
\def\PYZcb{\char`\}}
\def\PYZca{\char`\^}
\def\PYZam{\char`\&}
\def\PYZlt{\char`\<}
\def\PYZgt{\char`\>}
\def\PYZsh{\char`\#}
\def\PYZpc{\char`\%}
\def\PYZdl{\char`\$}
\def\PYZhy{\char`\-}
\def\PYZsq{\char`\'}
\def\PYZdq{\char`\"}
\def\PYZti{\char`\~}
% for compatibility with earlier versions
\def\PYZat{@}
\def\PYZlb{[}
\def\PYZrb{]}
\makeatother


    % For linebreaks inside Verbatim environment from package fancyvrb. 
    \makeatletter
        \newbox\Wrappedcontinuationbox 
        \newbox\Wrappedvisiblespacebox 
        \newcommand*\Wrappedvisiblespace {\textcolor{red}{\textvisiblespace}} 
        \newcommand*\Wrappedcontinuationsymbol {\textcolor{red}{\llap{\tiny$\m@th\hookrightarrow$}}} 
        \newcommand*\Wrappedcontinuationindent {3ex } 
        \newcommand*\Wrappedafterbreak {\kern\Wrappedcontinuationindent\copy\Wrappedcontinuationbox} 
        % Take advantage of the already applied Pygments mark-up to insert 
        % potential linebreaks for TeX processing. 
        %        {, <, #, %, $, ' and ": go to next line. 
        %        _, }, ^, &, >, - and ~: stay at end of broken line. 
        % Use of \textquotesingle for straight quote. 
        \newcommand*\Wrappedbreaksatspecials {% 
            \def\PYGZus{\discretionary{\char`\_}{\Wrappedafterbreak}{\char`\_}}% 
            \def\PYGZob{\discretionary{}{\Wrappedafterbreak\char`\{}{\char`\{}}% 
            \def\PYGZcb{\discretionary{\char`\}}{\Wrappedafterbreak}{\char`\}}}% 
            \def\PYGZca{\discretionary{\char`\^}{\Wrappedafterbreak}{\char`\^}}% 
            \def\PYGZam{\discretionary{\char`\&}{\Wrappedafterbreak}{\char`\&}}% 
            \def\PYGZlt{\discretionary{}{\Wrappedafterbreak\char`\<}{\char`\<}}% 
            \def\PYGZgt{\discretionary{\char`\>}{\Wrappedafterbreak}{\char`\>}}% 
            \def\PYGZsh{\discretionary{}{\Wrappedafterbreak\char`\#}{\char`\#}}% 
            \def\PYGZpc{\discretionary{}{\Wrappedafterbreak\char`\%}{\char`\%}}% 
            \def\PYGZdl{\discretionary{}{\Wrappedafterbreak\char`\$}{\char`\$}}% 
            \def\PYGZhy{\discretionary{\char`\-}{\Wrappedafterbreak}{\char`\-}}% 
            \def\PYGZsq{\discretionary{}{\Wrappedafterbreak\textquotesingle}{\textquotesingle}}% 
            \def\PYGZdq{\discretionary{}{\Wrappedafterbreak\char`\"}{\char`\"}}% 
            \def\PYGZti{\discretionary{\char`\~}{\Wrappedafterbreak}{\char`\~}}% 
        } 
        % Some characters . , ; ? ! / are not pygmentized. 
        % This macro makes them "active" and they will insert potential linebreaks 
        \newcommand*\Wrappedbreaksatpunct {% 
            \lccode`\~`\.\lowercase{\def~}{\discretionary{\hbox{\char`\.}}{\Wrappedafterbreak}{\hbox{\char`\.}}}% 
            \lccode`\~`\,\lowercase{\def~}{\discretionary{\hbox{\char`\,}}{\Wrappedafterbreak}{\hbox{\char`\,}}}% 
            \lccode`\~`\;\lowercase{\def~}{\discretionary{\hbox{\char`\;}}{\Wrappedafterbreak}{\hbox{\char`\;}}}% 
            \lccode`\~`\:\lowercase{\def~}{\discretionary{\hbox{\char`\:}}{\Wrappedafterbreak}{\hbox{\char`\:}}}% 
            \lccode`\~`\?\lowercase{\def~}{\discretionary{\hbox{\char`\?}}{\Wrappedafterbreak}{\hbox{\char`\?}}}% 
            \lccode`\~`\!\lowercase{\def~}{\discretionary{\hbox{\char`\!}}{\Wrappedafterbreak}{\hbox{\char`\!}}}% 
            \lccode`\~`\/\lowercase{\def~}{\discretionary{\hbox{\char`\/}}{\Wrappedafterbreak}{\hbox{\char`\/}}}% 
            \catcode`\.\active
            \catcode`\,\active 
            \catcode`\;\active
            \catcode`\:\active
            \catcode`\?\active
            \catcode`\!\active
            \catcode`\/\active 
            \lccode`\~`\~ 	
        }
    \makeatother

    \let\OriginalVerbatim=\Verbatim
    \makeatletter
    \renewcommand{\Verbatim}[1][1]{%
        %\parskip\z@skip
        \sbox\Wrappedcontinuationbox {\Wrappedcontinuationsymbol}%
        \sbox\Wrappedvisiblespacebox {\FV@SetupFont\Wrappedvisiblespace}%
        \def\FancyVerbFormatLine ##1{\hsize\linewidth
            \vtop{\raggedright\hyphenpenalty\z@\exhyphenpenalty\z@
                \doublehyphendemerits\z@\finalhyphendemerits\z@
                \strut ##1\strut}%
        }%
        % If the linebreak is at a space, the latter will be displayed as visible
        % space at end of first line, and a continuation symbol starts next line.
        % Stretch/shrink are however usually zero for typewriter font.
        \def\FV@Space {%
            \nobreak\hskip\z@ plus\fontdimen3\font minus\fontdimen4\font
            \discretionary{\copy\Wrappedvisiblespacebox}{\Wrappedafterbreak}
            {\kern\fontdimen2\font}%
        }%
        
        % Allow breaks at special characters using \PYG... macros.
        \Wrappedbreaksatspecials
        % Breaks at punctuation characters . , ; ? ! and / need catcode=\active 	
        \OriginalVerbatim[#1,codes*=\Wrappedbreaksatpunct]%
    }
    \makeatother

    % Exact colors from NB
    \definecolor{incolor}{HTML}{303F9F}
    \definecolor{outcolor}{HTML}{D84315}
    \definecolor{cellborder}{HTML}{CFCFCF}
    \definecolor{cellbackground}{HTML}{F7F7F7}
    
    % prompt
    \makeatletter
    \newcommand{\boxspacing}{\kern\kvtcb@left@rule\kern\kvtcb@boxsep}
    \makeatother
    \newcommand{\prompt}[4]{
        \ttfamily\llap{{\color{#2}[#3]:\hspace{3pt}#4}}\vspace{-\baselineskip}
    }
    

    
    % Prevent overflowing lines due to hard-to-break entities
    \sloppy 
    % Setup hyperref package
    \hypersetup{
      breaklinks=true,  % so long urls are correctly broken across lines
      colorlinks=true,
      urlcolor=urlcolor,
      linkcolor=linkcolor,
      citecolor=citecolor,
      }
    % Slightly bigger margins than the latex defaults
    
    \geometry{verbose,tmargin=1in,bmargin=1in,lmargin=1in,rmargin=1in}
    
    

\begin{document}
    
    \maketitle
    
    

    
    \hypertarget{ux440ux430ux437ux43bux438ux447ux43dux44bux435-ux43cux435ux442ux43eux434ux44b-ux43eux43fux442ux438ux43cux438ux437ux430ux446ux438ux438-ux434ux43bux44f-ux440ux435ux448ux435ux43dux438ux44f-ux43bux438ux43dux435ux439ux43dux43eux439-ux440ux435ux433ux440ux435ux441ux441ux438ux438}{%
\section{Различные методы оптимизации для решения линейной
регрессии}\label{ux440ux430ux437ux43bux438ux447ux43dux44bux435-ux43cux435ux442ux43eux434ux44b-ux43eux43fux442ux438ux43cux438ux437ux430ux446ux438ux438-ux434ux43bux44f-ux440ux435ux448ux435ux43dux438ux44f-ux43bux438ux43dux435ux439ux43dux43eux439-ux440ux435ux433ux440ux435ux441ux441ux438ux438}}

Седых Иван Дмитриевич БПМ185

    \hypertarget{ux438ux43cux43fux43eux440ux442-ux43dux443ux436ux43dux44bux445-ux431ux438ux431ux43bux438ux43eux442ux435ux43a}{%
\subsection{Импорт нужных
библиотек}\label{ux438ux43cux43fux43eux440ux442-ux43dux443ux436ux43dux44bux445-ux431ux438ux431ux43bux438ux43eux442ux435ux43a}}

    \begin{tcolorbox}[breakable, size=fbox, boxrule=1pt, pad at break*=1mm,colback=cellbackground, colframe=cellborder]
\prompt{In}{incolor}{1}{\boxspacing}
\begin{Verbatim}[commandchars=\\\{\}]
\PY{k+kn}{from} \PY{n+nn}{sklearn}\PY{n+nn}{.}\PY{n+nn}{datasets} \PY{k+kn}{import} \PY{n}{make\PYZus{}regression}
\PY{k+kn}{from} \PY{n+nn}{sklearn}\PY{n+nn}{.}\PY{n+nn}{linear\PYZus{}model} \PY{k+kn}{import} \PY{n}{LinearRegression}
\PY{k+kn}{from} \PY{n+nn}{matplotlib} \PY{k+kn}{import} \PY{n}{pyplot} \PY{k}{as} \PY{n}{plt}
\PY{k+kn}{import} \PY{n+nn}{numpy} \PY{k}{as} \PY{n+nn}{np}
\PY{k+kn}{from} \PY{n+nn}{mpl\PYZus{}toolkits}\PY{n+nn}{.}\PY{n+nn}{mplot3d} \PY{k+kn}{import} \PY{n}{Axes3D}
\PY{k+kn}{from} \PY{n+nn}{matplotlib} \PY{k+kn}{import} \PY{n}{cm}
\PY{o}{\PYZpc{}}\PY{k}{matplotlib} inline
\end{Verbatim}
\end{tcolorbox}

    \hypertarget{ux441ux433ux435ux43dux435ux440ux438ux440ux443ux435ux43c-ux434ux430ux43dux43dux44bux435}{%
\subsection{Сгенерируем
данные}\label{ux441ux433ux435ux43dux435ux440ux438ux440ux443ux435ux43c-ux434ux430ux43dux43dux44bux435}}

Генерируются они по следующему правилу: \[
y = w_1x+w_0+\epsilon
\] Где:\\
\(w\) - коэффициенты, в нахождении которых и заключается задача\\
\(\epsilon\) - гауссовский шум с \(\sigma=\) \texttt{noise}, \(\mu=0\)

    \begin{tcolorbox}[breakable, size=fbox, boxrule=1pt, pad at break*=1mm,colback=cellbackground, colframe=cellborder]
\prompt{In}{incolor}{2}{\boxspacing}
\begin{Verbatim}[commandchars=\\\{\}]
\PY{n}{b} \PY{o}{=} \PY{l+m+mi}{20}
\PY{n}{X}\PY{p}{,} \PY{n}{y}\PY{p}{,} \PY{n}{coefs} \PY{o}{=} \PY{n}{make\PYZus{}regression}\PY{p}{(} 
                        \PY{n}{n\PYZus{}samples}\PY{o}{=}\PY{l+m+mi}{100}\PY{p}{,} \PY{c+c1}{\PYZsh{}названия парметров }
                        \PY{n}{n\PYZus{}features}\PY{o}{=}\PY{l+m+mi}{1}\PY{p}{,}
                        \PY{n}{bias}\PY{o}{=}\PY{n}{b}\PY{p}{,}
                        \PY{n}{noise}\PY{o}{=}\PY{l+m+mi}{5}\PY{p}{,}
                        \PY{n}{coef}\PY{o}{=}\PY{k+kc}{True}\PY{p}{,}
                        \PY{n}{random\PYZus{}state}\PY{o}{=}\PY{l+m+mi}{42}
                        \PY{p}{)}

\PY{c+c1}{\PYZsh{} Приделаем справа столбец из единиц, чтобы формула регрессии приняла вид: y=aX}
\PY{n}{X} \PY{o}{=} \PY{n}{np}\PY{o}{.}\PY{n}{hstack}\PY{p}{(}\PY{p}{(}\PY{n}{X}\PY{p}{,}\PY{n}{np}\PY{o}{.}\PY{n}{ones}\PY{p}{(}\PY{p}{(}\PY{n}{X}\PY{o}{.}\PY{n}{shape}\PY{p}{[}\PY{l+m+mi}{0}\PY{p}{]}\PY{p}{,}\PY{l+m+mi}{1}\PY{p}{)}\PY{p}{)}\PY{p}{)}\PY{p}{)}

\PY{n}{coefs} \PY{o}{=} \PY{n}{np}\PY{o}{.}\PY{n}{array}\PY{p}{(}\PY{p}{[}\PY{n}{coefs}\PY{p}{,} \PY{n}{b}\PY{p}{]}\PY{p}{)}
\PY{n+nb}{print}\PY{p}{(}\PY{l+s+s1}{\PYZsq{}}\PY{l+s+s1}{искомые коэффициенты:}\PY{l+s+s1}{\PYZsq{}}\PY{p}{)}
\PY{n+nb}{print}\PY{p}{(}\PY{n}{coefs}\PY{p}{)}
\end{Verbatim}
\end{tcolorbox}

    \begin{Verbatim}[commandchars=\\\{\}]
искомые коэффициенты:
[41.74110031 20.        ]
    \end{Verbatim}

    \begin{tcolorbox}[breakable, size=fbox, boxrule=1pt, pad at break*=1mm,colback=cellbackground, colframe=cellborder]
\prompt{In}{incolor}{3}{\boxspacing}
\begin{Verbatim}[commandchars=\\\{\}]
\PY{n}{plt}\PY{o}{.}\PY{n}{scatter}\PY{p}{(}\PY{n}{X}\PY{p}{[}\PY{p}{:}\PY{p}{,}\PY{p}{:}\PY{o}{\PYZhy{}}\PY{l+m+mi}{1}\PY{p}{]}\PY{p}{,}\PY{n}{y}\PY{p}{)}
\PY{n}{plt}\PY{o}{.}\PY{n}{show}\PY{p}{(}\PY{p}{)}
\end{Verbatim}
\end{tcolorbox}

    \begin{center}
    \adjustimage{max size={0.9\linewidth}{0.9\paperheight}}{homework_files/homework_5_0.pdf}
    \end{center}
    { \hspace*{\fill} \\}
    
    \hypertarget{ux43eux43fux440ux435ux434ux435ux43bux438ux43c-ux444ux443ux43dux43aux446ux438ux44e-ux43aux43eux442ux43eux440ux443ux44e-ux431ux443ux434ux435ux43c-ux43eux43fux442ux438ux43cux438ux437ux438ux440ux43eux432ux430ux442ux44c}{%
\subsection{Определим функцию которую будем
оптимизировать}\label{ux43eux43fux440ux435ux434ux435ux43bux438ux43c-ux444ux443ux43dux43aux446ux438ux44e-ux43aux43eux442ux43eux440ux443ux44e-ux431ux443ux434ux435ux43c-ux43eux43fux442ux438ux43cux438ux437ux438ux440ux43eux432ux430ux442ux44c}}

Стандарт для задачи регресии - \(MeanSquaredError (MSE)\) \[
MSE(y,\hat{y}) = \frac{1}{l}\sum_{i=1}^{l} (y_i-\hat{y_i})^2 
\] или, для нашего случая:
\[MSE(X,y) = \frac{1}{l}\sum_{i=1}^{l} (y_i-x_iw)^2\]

    \begin{tcolorbox}[breakable, size=fbox, boxrule=1pt, pad at break*=1mm,colback=cellbackground, colframe=cellborder]
\prompt{In}{incolor}{4}{\boxspacing}
\begin{Verbatim}[commandchars=\\\{\}]
\PY{k}{def} \PY{n+nf}{mse}\PY{p}{(}\PY{n}{w0}\PY{p}{,}\PY{n}{w1}\PY{p}{)}\PY{p}{:}
    \PY{k}{return} \PY{n}{np}\PY{o}{.}\PY{n}{mean}\PY{p}{(}
        \PY{n}{np}\PY{o}{.}\PY{n}{power}\PY{p}{(}
            \PY{n}{X}\PY{n+nd}{@np}\PY{o}{.}\PY{n}{array}\PY{p}{(}\PY{p}{[}\PY{n}{w1}\PY{p}{,}\PY{n}{w0}\PY{p}{]}\PY{p}{)} \PY{o}{\PYZhy{}} \PY{n}{y}\PY{p}{,} \PY{l+m+mi}{2}
        \PY{p}{)}
    \PY{p}{)}
\end{Verbatim}
\end{tcolorbox}

    \hypertarget{ux43fux43eux441ux442ux440ux43eux438ux43c-ux433ux440ux430ux444ux438ux43a-ux444ux443ux43dux43aux446ux438ux438-ux43fux43eux442ux435ux440ux44c}{%
\subsection{Построим график функции
потерь}\label{ux43fux43eux441ux442ux440ux43eux438ux43c-ux433ux440ux430ux444ux438ux43a-ux444ux443ux43dux43aux446ux438ux438-ux43fux43eux442ux435ux440ux44c}}

Видно, что она гладкая выпуклая и вообще хорошая с одним минимумом,
который является глобальным

    \begin{tcolorbox}[breakable, size=fbox, boxrule=1pt, pad at break*=1mm,colback=cellbackground, colframe=cellborder]
\prompt{In}{incolor}{5}{\boxspacing}
\begin{Verbatim}[commandchars=\\\{\}]
\PY{n}{fig} \PY{o}{=} \PY{n}{plt}\PY{o}{.}\PY{n}{figure}\PY{p}{(}\PY{n}{figsize}\PY{o}{=}\PY{p}{(}\PY{l+m+mi}{8}\PY{p}{,}\PY{l+m+mi}{8}\PY{p}{)}\PY{p}{)}
\PY{n}{ax} \PY{o}{=} \PY{n}{fig}\PY{o}{.}\PY{n}{gca}\PY{p}{(}\PY{n}{projection}\PY{o}{=}\PY{l+s+s1}{\PYZsq{}}\PY{l+s+s1}{3d}\PY{l+s+s1}{\PYZsq{}}\PY{p}{)}

\PY{c+c1}{\PYZsh{} Make data.}
\PY{n}{xx} \PY{o}{=} \PY{n}{np}\PY{o}{.}\PY{n}{arange}\PY{p}{(}\PY{l+m+mi}{10}\PY{p}{,} \PY{l+m+mi}{35}\PY{p}{,} \PY{l+m+mi}{1}\PY{p}{)}
\PY{n}{yy} \PY{o}{=} \PY{n}{np}\PY{o}{.}\PY{n}{arange}\PY{p}{(}\PY{l+m+mi}{30}\PY{p}{,} \PY{l+m+mi}{55}\PY{p}{,} \PY{l+m+mi}{1}\PY{p}{)}
\PY{n}{xx}\PY{p}{,} \PY{n}{yy} \PY{o}{=} \PY{n}{np}\PY{o}{.}\PY{n}{meshgrid}\PY{p}{(}\PY{n}{xx}\PY{p}{,} \PY{n}{yy}\PY{p}{)}
\PY{n}{Z} \PY{o}{=} \PY{n}{np}\PY{o}{.}\PY{n}{vectorize}\PY{p}{(}\PY{n}{mse}\PY{p}{)}\PY{p}{(}\PY{n}{xx}\PY{p}{,}\PY{n}{yy}\PY{p}{)}

\PY{c+c1}{\PYZsh{} Plot the surface.}
\PY{n}{surf} \PY{o}{=} \PY{n}{ax}\PY{o}{.}\PY{n}{plot\PYZus{}surface}\PY{p}{(}\PY{n}{xx}\PY{p}{,} \PY{n}{yy}\PY{p}{,} \PY{n}{Z}\PY{p}{,} \PY{n}{cmap}\PY{o}{=}\PY{n}{cm}\PY{o}{.}\PY{n}{coolwarm}\PY{p}{,}
                       \PY{n}{linewidth}\PY{o}{=}\PY{l+m+mi}{0}\PY{p}{,} \PY{n}{antialiased}\PY{o}{=}\PY{k+kc}{False}\PY{p}{)}
\PY{n}{plt}\PY{o}{.}\PY{n}{xlabel}\PY{p}{(}\PY{l+s+s1}{\PYZsq{}}\PY{l+s+s1}{\PYZdl{}w\PYZus{}0\PYZdl{}}\PY{l+s+s1}{\PYZsq{}}\PY{p}{)}
\PY{n}{plt}\PY{o}{.}\PY{n}{ylabel}\PY{p}{(}\PY{l+s+s1}{\PYZsq{}}\PY{l+s+s1}{\PYZdl{}w\PYZus{}1\PYZdl{}}\PY{l+s+s1}{\PYZsq{}}\PY{p}{)}
\PY{n}{plt}\PY{o}{.}\PY{n}{show}\PY{p}{(}\PY{p}{)}

\PY{n}{fig}\PY{p}{,} \PY{n}{ax} \PY{o}{=} \PY{n}{plt}\PY{o}{.}\PY{n}{subplots}\PY{p}{(}\PY{n}{figsize}\PY{o}{=}\PY{p}{(}\PY{l+m+mi}{8}\PY{p}{,}\PY{l+m+mi}{8}\PY{p}{)}\PY{p}{)}

\PY{n}{z\PYZus{}min}\PY{p}{,} \PY{n}{z\PYZus{}max} \PY{o}{=} \PY{l+m+mi}{0}\PY{p}{,} \PY{n}{np}\PY{o}{.}\PY{n}{abs}\PY{p}{(}\PY{n}{Z}\PY{p}{)}\PY{o}{.}\PY{n}{max}\PY{p}{(}\PY{p}{)}
\PY{n}{c} \PY{o}{=} \PY{n}{ax}\PY{o}{.}\PY{n}{pcolormesh}\PY{p}{(}\PY{n}{xx}\PY{p}{,} \PY{n}{yy}\PY{p}{,} \PY{n}{Z}\PY{p}{,} \PY{n}{cmap}\PY{o}{=}\PY{n}{cm}\PY{o}{.}\PY{n}{coolwarm}\PY{p}{,} \PY{n}{vmin}\PY{o}{=}\PY{n}{z\PYZus{}min}\PY{p}{,} \PY{n}{vmax}\PY{o}{=}\PY{n}{z\PYZus{}max}\PY{p}{)}
\PY{n}{ax}\PY{o}{.}\PY{n}{set\PYZus{}title}\PY{p}{(}\PY{l+s+s1}{\PYZsq{}}\PY{l+s+s1}{Вид сверху}\PY{l+s+s1}{\PYZsq{}}\PY{p}{)}
\PY{n}{ax}\PY{o}{.}\PY{n}{axis}\PY{p}{(}\PY{p}{[}\PY{n}{xx}\PY{o}{.}\PY{n}{min}\PY{p}{(}\PY{p}{)}\PY{p}{,} \PY{n}{xx}\PY{o}{.}\PY{n}{max}\PY{p}{(}\PY{p}{)}\PY{p}{,} \PY{n}{yy}\PY{o}{.}\PY{n}{min}\PY{p}{(}\PY{p}{)}\PY{p}{,} \PY{n}{yy}\PY{o}{.}\PY{n}{max}\PY{p}{(}\PY{p}{)}\PY{p}{]}\PY{p}{)}
\PY{n}{fig}\PY{o}{.}\PY{n}{colorbar}\PY{p}{(}\PY{n}{c}\PY{p}{,} \PY{n}{ax}\PY{o}{=}\PY{n}{ax}\PY{p}{)}
\PY{n}{plt}\PY{o}{.}\PY{n}{xlabel}\PY{p}{(}\PY{l+s+s1}{\PYZsq{}}\PY{l+s+s1}{\PYZdl{}w\PYZus{}0\PYZdl{}}\PY{l+s+s1}{\PYZsq{}}\PY{p}{)}
\PY{n}{plt}\PY{o}{.}\PY{n}{ylabel}\PY{p}{(}\PY{l+s+s1}{\PYZsq{}}\PY{l+s+s1}{\PYZdl{}w\PYZus{}1\PYZdl{}}\PY{l+s+s1}{\PYZsq{}}\PY{p}{)}
\PY{n}{plt}\PY{o}{.}\PY{n}{show}\PY{p}{(}\PY{p}{)}
\end{Verbatim}
\end{tcolorbox}

    \begin{center}
    \adjustimage{max size={0.9\linewidth}{0.9\paperheight}}{homework_files/homework_9_0.pdf}
    \end{center}
    { \hspace*{\fill} \\}
    
    \begin{center}
    \adjustimage{max size={0.9\linewidth}{0.9\paperheight}}{homework_files/homework_9_1.pdf}
    \end{center}
    { \hspace*{\fill} \\}
    
    \hypertarget{ux430ux43dux430ux43bux438ux442ux438ux447ux435ux441ux43aux43eux435-ux440ux435ux448ux435ux43dux438ux435}{%
\subsection{Аналитическое
решение}\label{ux430ux43dux430ux43bux438ux442ux438ux447ux435ux441ux43aux43eux435-ux440ux435ux448ux435ux43dux438ux435}}

Для этой задачи можно получить аналитическое решение явно,
продифференцировав функционал ошибки (уберем множитель \(\frac{1}{l}\),
минимум от этого не изменится). \[
l \cdot MSE = (y-Xw)^T(y-Xw)
\] Дифференцируя эту функцию по вектору параметров \(w\) и приравняв
производные к нулю, \[
\frac{\partial MSE}{\partial w} = -2 X^T y+2X^TXw =0 
\] получим систему уравнений \[
(X^TX)w=X^Ty
\] Решение которой:\[w =(X^TX)^{-1}X^Ty \]

    \begin{tcolorbox}[breakable, size=fbox, boxrule=1pt, pad at break*=1mm,colback=cellbackground, colframe=cellborder]
\prompt{In}{incolor}{6}{\boxspacing}
\begin{Verbatim}[commandchars=\\\{\}]
\PY{n}{w} \PY{o}{=} \PY{n}{np}\PY{o}{.}\PY{n}{linalg}\PY{o}{.}\PY{n}{inv}\PY{p}{(}\PY{n}{X}\PY{o}{.}\PY{n}{T} \PY{o}{@} \PY{n}{X}\PY{p}{)} \PY{o}{@} \PY{n}{X}\PY{o}{.}\PY{n}{T} \PY{o}{@} \PY{n}{y}
\PY{n+nb}{print}\PY{p}{(}\PY{l+s+s1}{\PYZsq{}}\PY{l+s+s1}{полученное решение:}\PY{l+s+s1}{\PYZsq{}}\PY{p}{,} \PY{n}{w}\PY{p}{)}
\PY{n+nb}{print}\PY{p}{(}\PY{l+s+s1}{\PYZsq{}}\PY{l+s+s1}{MSE = }\PY{l+s+si}{\PYZob{}\PYZcb{}}\PY{l+s+s1}{\PYZsq{}}\PY{o}{.}\PY{n}{format}\PY{p}{(}\PY{n}{mse}\PY{p}{(}\PY{o}{*}\PY{n}{w}\PY{p}{[}\PY{p}{:}\PY{p}{:}\PY{o}{\PYZhy{}}\PY{l+m+mi}{1}\PY{p}{]}\PY{p}{)}\PY{p}{)}\PY{p}{)}
\end{Verbatim}
\end{tcolorbox}

    \begin{Verbatim}[commandchars=\\\{\}]
полученное решение: [43.08913515 20.58255766]
MSE = 19.513562605309758
    \end{Verbatim}

    Зачем тогда, собственно, нужны другие методы оптимизации?\\
Дело в том, что взятие обратной матрицы может быть проблематично (это
сложная операция - \(O(n^3)\), если без ухищрений) или вообще не
определено, если матрица вырождена. К тому же, если матрица \(X\) -
большая, то быть может есть смысл использовать не все её значения, что
неминуемо приводит нас к следующему методу

    \hypertarget{ux441ux442ux43eux445ux430ux441ux442ux438ux447ux435ux441ux43aux438ux439-ux433ux440ux430ux434ux438ux435ux43dux442ux43dux44bux439-ux441ux43fux443ux441ux43a-sgd}{%
\subsection{Стохастический градиентный спуск
(SGD)}\label{ux441ux442ux43eux445ux430ux441ux442ux438ux447ux435ux441ux43aux438ux439-ux433ux440ux430ux434ux438ux435ux43dux442ux43dux44bux439-ux441ux43fux443ux441ux43a-sgd}}

\textbf{Небольшая справка об обычном градиентном спуске:}\\
\emph{При стандартном градиентном спуске для корректировки параметров
модели используется градиент. Градиент обычно считается как сумма
градиентов, вызванных каждым элементом обучения. Вектор параметров
изменяется в направлении антиградиента с заданным шагом. Поэтому
стандартному градиентному спуску требуется один проход по обучающим
данным до того, как он сможет менять параметры.}

\textbf{Объяснении на пальцах:} Представим, что наша функция потерь
(трехмерный график выше -) \_ это гора, а мы стоим в случайном месте на
лыжах. Теперь катимся вниз(вниз - это по направлению антиградиента).
Через какое-то время мы скатимся вниз - это и будет локальный минимум.

\textbf{Теперь о стохастическом градиентном спуске:}\\
\emph{При стохастическом (или «оперативном») градиентном спуске значение
градиента аппроксимируются градиентом функции стоимости (потерь, в нашем
случае MSE), вычисленном только на одном элементе обучения. Затем
параметры изменяются пропорционально приближенному градиенту. Таким
образом параметры модели изменяются после каждого объекта обучения. Для
больших массивов данных стохастический градиентный спуск может дать
значительное преимущество в скорости по сравнению со стандартным
градиентным спуском.}

Формула для шага градиентного спуска в нашем случае выглядит так:
\[w = w - \eta \nabla MSE \] или \[
w_i = w_i - \eta \frac{\partial MSE}{\partial w_i}
\]
\href{http://www.machinelearning.ru/wiki/index.php?title=\%D0\%9C\%D0\%B5\%D1\%82\%D0\%BE\%D0\%B4_\%D1\%81\%D1\%82\%D0\%BE\%D1\%85\%D0\%B0\%D1\%81\%D1\%82\%D0\%B8\%D1\%87\%D0\%B5\%D1\%81\%D0\%BA\%D0\%BE\%D0\%B3\%D0\%BE_\%D0\%B3\%D1\%80\%D0\%B0\%D0\%B4\%D0\%B8\%D0\%B5\%D0\%BD\%D1\%82\%D0\%B0}{source}

    \begin{tcolorbox}[breakable, size=fbox, boxrule=1pt, pad at break*=1mm,colback=cellbackground, colframe=cellborder]
\prompt{In}{incolor}{7}{\boxspacing}
\begin{Verbatim}[commandchars=\\\{\}]
\PY{o}{\PYZpc{}\PYZpc{}time}
\PY{n}{history} \PY{o}{=} \PY{n}{np}\PY{o}{.}\PY{n}{array}\PY{p}{(}\PY{p}{[}\PY{p}{]}\PY{p}{)}
\PY{n}{w} \PY{o}{=} \PY{n}{np}\PY{o}{.}\PY{n}{zeros}\PY{p}{(}\PY{n}{shape}\PY{o}{=}\PY{n}{X}\PY{o}{.}\PY{n}{shape}\PY{p}{[}\PY{l+m+mi}{1}\PY{p}{]}\PY{p}{)}
\PY{n}{lr} \PY{o}{=} \PY{l+m+mf}{1e\PYZhy{}3}

\PY{n}{steps} \PY{o}{=} \PY{l+m+mi}{10000} 
\PY{k}{for} \PY{n}{\PYZus{}} \PY{o+ow}{in} \PY{n+nb}{range}\PY{p}{(}\PY{n}{steps}\PY{p}{)}\PY{p}{:}
     \PY{n}{random\PYZus{}id} \PY{o}{=} \PY{n}{np}\PY{o}{.}\PY{n}{random}\PY{o}{.}\PY{n}{choice}\PY{p}{(}\PY{n}{X}\PY{o}{.}\PY{n}{shape}\PY{p}{[}\PY{l+m+mi}{0}\PY{p}{]}\PY{p}{)}
     \PY{n}{xi} \PY{o}{=} \PY{n}{X}\PY{p}{[}\PY{n}{random\PYZus{}id}\PY{p}{]}
     \PY{n}{yi} \PY{o}{=} \PY{n}{y}\PY{p}{[}\PY{n}{random\PYZus{}id}\PY{p}{]}
     \PY{n}{grad\PYZus{}w0} \PY{o}{=} \PY{o}{\PYZhy{}}\PY{l+m+mi}{2} \PY{o}{*} \PY{p}{(}\PY{n}{yi} \PY{o}{\PYZhy{}} \PY{n}{xi} \PY{o}{@} \PY{n}{w}\PY{p}{[}\PY{p}{:}\PY{p}{:}\PY{o}{\PYZhy{}}\PY{l+m+mi}{1}\PY{p}{]}\PY{p}{)}
     \PY{n}{grad\PYZus{}w1} \PY{o}{=} \PY{o}{\PYZhy{}}\PY{l+m+mi}{2} \PY{o}{*} \PY{n}{xi}\PY{p}{[}\PY{l+m+mi}{0}\PY{p}{]}\PY{o}{*} \PY{p}{(}\PY{n}{yi} \PY{o}{\PYZhy{}} \PY{n}{xi} \PY{o}{@} \PY{n}{w}\PY{p}{[}\PY{p}{:}\PY{p}{:}\PY{o}{\PYZhy{}}\PY{l+m+mi}{1}\PY{p}{]}\PY{p}{)}
     \PY{n}{w}\PY{p}{[}\PY{l+m+mi}{0}\PY{p}{]} \PY{o}{=} \PY{n}{w}\PY{p}{[}\PY{l+m+mi}{0}\PY{p}{]} \PY{o}{\PYZhy{}} \PY{n}{lr} \PY{o}{*} \PY{n}{grad\PYZus{}w0}
     \PY{n}{w}\PY{p}{[}\PY{l+m+mi}{1}\PY{p}{]} \PY{o}{=} \PY{n}{w}\PY{p}{[}\PY{l+m+mi}{1}\PY{p}{]} \PY{o}{\PYZhy{}} \PY{n}{lr} \PY{o}{*} \PY{n}{grad\PYZus{}w1}
     \PY{n}{history} \PY{o}{=} \PY{n}{np}\PY{o}{.}\PY{n}{append}\PY{p}{(}\PY{n}{history}\PY{p}{,} \PY{n}{mse}\PY{p}{(}\PY{o}{*}\PY{n}{w}\PY{p}{)}\PY{p}{)}
\PY{n+nb}{print}\PY{p}{(}\PY{l+s+s1}{\PYZsq{}}\PY{l+s+s1}{полученное решение:}\PY{l+s+s1}{\PYZsq{}}\PY{p}{,}\PY{n}{w}\PY{p}{[}\PY{p}{:}\PY{p}{:}\PY{o}{\PYZhy{}}\PY{l+m+mi}{1}\PY{p}{]}\PY{p}{)}
\PY{n+nb}{print}\PY{p}{(}\PY{l+s+s1}{\PYZsq{}}\PY{l+s+s1}{MSE = }\PY{l+s+si}{\PYZob{}\PYZcb{}}\PY{l+s+s1}{\PYZsq{}}\PY{o}{.}\PY{n}{format}\PY{p}{(}\PY{n}{history}\PY{p}{[}\PY{o}{\PYZhy{}}\PY{l+m+mi}{1}\PY{p}{]}\PY{p}{)}\PY{p}{)}
\PY{n}{plt}\PY{o}{.}\PY{n}{title}\PY{p}{(}\PY{l+s+s1}{\PYZsq{}}\PY{l+s+s1}{MSE}\PY{l+s+s1}{\PYZsq{}}\PY{p}{)}
\PY{n}{plt}\PY{o}{.}\PY{n}{xlabel}\PY{p}{(}\PY{l+s+s1}{\PYZsq{}}\PY{l+s+s1}{номер итерации}\PY{l+s+s1}{\PYZsq{}}\PY{p}{)}
\PY{n}{plt}\PY{o}{.}\PY{n}{plot}\PY{p}{(}\PY{n}{history}\PY{p}{)}
\PY{n}{plt}\PY{o}{.}\PY{n}{show}\PY{p}{(}\PY{p}{)}
\end{Verbatim}
\end{tcolorbox}

    \begin{Verbatim}[commandchars=\\\{\}]
полученное решение: [43.08308701 20.57648821]
MSE = 19.513622082287583
    \end{Verbatim}

    \begin{center}
    \adjustimage{max size={0.9\linewidth}{0.9\paperheight}}{homework_files/homework_14_1.pdf}
    \end{center}
    { \hspace*{\fill} \\}
    
    \begin{Verbatim}[commandchars=\\\{\}]
CPU times: user 1.09 s, sys: 28.9 ms, total: 1.12 s
Wall time: 1.29 s
    \end{Verbatim}

    УРА! Алгоритм сошелся к минимуму! Решение похоже на полученное
аналитически.\\
Хороший метод, но он имеет свои недостатки:\\
* Так как алгоритм сходится к локальному минимуму, то, возможно, что
найденный минимум не глобальный * Может не сойтись

    \hypertarget{ux43aux43bux430ux441ux441ux438ux447ux435ux441ux43aux438ux439-ux433ux440ux430ux434ux438ux435ux43dux442ux43dux44bux439-ux441ux43fux443ux441ux43a}{%
\subsection{Классический градиентный
спуск}\label{ux43aux43bux430ux441ux441ux438ux447ux435ux441ux43aux438ux439-ux433ux440ux430ux434ux438ux435ux43dux442ux43dux44bux439-ux441ux43fux443ux441ux43a}}

Легко можно трансформировать стохастический градиентный спуск в
классический

    \begin{tcolorbox}[breakable, size=fbox, boxrule=1pt, pad at break*=1mm,colback=cellbackground, colframe=cellborder]
\prompt{In}{incolor}{8}{\boxspacing}
\begin{Verbatim}[commandchars=\\\{\}]
\PY{o}{\PYZpc{}\PYZpc{}time}
\PY{n}{history} \PY{o}{=} \PY{n}{np}\PY{o}{.}\PY{n}{array}\PY{p}{(}\PY{p}{[}\PY{p}{]}\PY{p}{)}
\PY{n}{w} \PY{o}{=} \PY{n}{np}\PY{o}{.}\PY{n}{zeros}\PY{p}{(}\PY{n}{shape}\PY{o}{=}\PY{n}{X}\PY{o}{.}\PY{n}{shape}\PY{p}{[}\PY{l+m+mi}{1}\PY{p}{]}\PY{p}{)}
\PY{n}{lr} \PY{o}{=} \PY{l+m+mf}{1e\PYZhy{}3}

\PY{n}{steps} \PY{o}{=} \PY{l+m+mi}{10000}
\PY{k}{for} \PY{n}{\PYZus{}} \PY{o+ow}{in} \PY{n+nb}{range}\PY{p}{(}\PY{n}{steps}\PY{p}{)}\PY{p}{:}
     \PY{n}{grad\PYZus{}w0} \PY{o}{=} \PY{o}{\PYZhy{}}\PY{l+m+mi}{2} \PY{o}{*} \PY{p}{(}\PY{n}{y} \PY{o}{\PYZhy{}} \PY{n}{X} \PY{o}{@} \PY{n}{w}\PY{p}{[}\PY{p}{:}\PY{p}{:}\PY{o}{\PYZhy{}}\PY{l+m+mi}{1}\PY{p}{]}\PY{p}{)}\PY{o}{.}\PY{n}{mean}\PY{p}{(}\PY{p}{)}
     \PY{n}{grad\PYZus{}w1} \PY{o}{=} \PY{o}{\PYZhy{}}\PY{l+m+mi}{2} \PY{o}{*} \PY{p}{(}\PY{n}{X}\PY{p}{[}\PY{p}{:}\PY{p}{,}\PY{l+m+mi}{0}\PY{p}{]} \PY{o}{*} \PY{p}{(}\PY{n}{y} \PY{o}{\PYZhy{}} \PY{n}{X} \PY{o}{@} \PY{n}{w}\PY{p}{[}\PY{p}{:}\PY{p}{:}\PY{o}{\PYZhy{}}\PY{l+m+mi}{1}\PY{p}{]}\PY{p}{)}\PY{p}{)}\PY{o}{.}\PY{n}{mean}\PY{p}{(}\PY{p}{)}
     \PY{n}{w}\PY{p}{[}\PY{l+m+mi}{0}\PY{p}{]} \PY{o}{=} \PY{n}{w}\PY{p}{[}\PY{l+m+mi}{0}\PY{p}{]} \PY{o}{\PYZhy{}} \PY{n}{lr} \PY{o}{*} \PY{n}{grad\PYZus{}w0}
     \PY{n}{w}\PY{p}{[}\PY{l+m+mi}{1}\PY{p}{]} \PY{o}{=} \PY{n}{w}\PY{p}{[}\PY{l+m+mi}{1}\PY{p}{]} \PY{o}{\PYZhy{}} \PY{n}{lr} \PY{o}{*} \PY{n}{grad\PYZus{}w1}
     \PY{n}{history} \PY{o}{=} \PY{n}{np}\PY{o}{.}\PY{n}{append}\PY{p}{(}\PY{n}{history}\PY{p}{,} \PY{n}{mse}\PY{p}{(}\PY{o}{*}\PY{n}{w}\PY{p}{)}\PY{p}{)}
\PY{n+nb}{print}\PY{p}{(}\PY{l+s+s1}{\PYZsq{}}\PY{l+s+s1}{полученное решение:}\PY{l+s+s1}{\PYZsq{}}\PY{p}{,}\PY{n}{w}\PY{p}{[}\PY{p}{:}\PY{p}{:}\PY{o}{\PYZhy{}}\PY{l+m+mi}{1}\PY{p}{]}\PY{p}{)}
\PY{n+nb}{print}\PY{p}{(}\PY{l+s+s1}{\PYZsq{}}\PY{l+s+s1}{MSE = }\PY{l+s+si}{\PYZob{}\PYZcb{}}\PY{l+s+s1}{\PYZsq{}}\PY{o}{.}\PY{n}{format}\PY{p}{(}\PY{n}{history}\PY{p}{[}\PY{o}{\PYZhy{}}\PY{l+m+mi}{1}\PY{p}{]}\PY{p}{)}\PY{p}{)}
\PY{n}{plt}\PY{o}{.}\PY{n}{title}\PY{p}{(}\PY{l+s+s1}{\PYZsq{}}\PY{l+s+s1}{MSE}\PY{l+s+s1}{\PYZsq{}}\PY{p}{)}
\PY{n}{plt}\PY{o}{.}\PY{n}{xlabel}\PY{p}{(}\PY{l+s+s1}{\PYZsq{}}\PY{l+s+s1}{номер итерации}\PY{l+s+s1}{\PYZsq{}}\PY{p}{)}
\PY{n}{plt}\PY{o}{.}\PY{n}{plot}\PY{p}{(}\PY{n}{history}\PY{p}{)}
\PY{n}{plt}\PY{o}{.}\PY{n}{show}\PY{p}{(}\PY{p}{)}
\end{Verbatim}
\end{tcolorbox}

    \begin{Verbatim}[commandchars=\\\{\}]
полученное решение: [43.08912778 20.5825542 ]
MSE = 19.5135626053614
    \end{Verbatim}

    \begin{center}
    \adjustimage{max size={0.9\linewidth}{0.9\paperheight}}{homework_files/homework_17_1.pdf}
    \end{center}
    { \hspace*{\fill} \\}
    
    \begin{Verbatim}[commandchars=\\\{\}]
CPU times: user 1.04 s, sys: 25.1 ms, total: 1.06 s
Wall time: 1.15 s
    \end{Verbatim}

    Также в наше время постоянно придумываются новые методы, схожие с
градиентным спуском, типа Adam, RMSprop, Adadelta и прочие. Их
используют в глубоком обучении, где нужно эффективно оптимизировать
функции очень(\textbf{очень}) многих переменных. Оценить масштаб
происходящего можно в документации к разделу оптимизаторов библиотеки
\href{https://pytorch.org/docs/stable/optim.html}{Pytorch}.\\
Но моя работа не об этом, я сконценитрируюсь на более глубоком понимании
классических методов.

    \hypertarget{ux43cux435ux442ux43eux434-ux43dux44cux44eux442ux43eux43dux430}{%
\subsection{Метод
Ньютона}\label{ux43cux435ux442ux43eux434-ux43dux44cux44eux442ux43eux43dux430}}

Формула дла шага метода Ньютона выглдядит так: \[
w = w - H^{-1}(w)\nabla MSE(w)
\] где \(H\) - гессиан \(MSE\) \[
\nabla MSE(w) = \begin{pmatrix}
 -2 \frac{1}{l} \sum_{i=1}^l y_i - x_iw \\
 -2 \frac{1}{l} \sum_{i=1}^l x_i(y_i - x_iw) 
  \end{pmatrix}
\] \[
H = \begin{pmatrix}
2 & 2 \frac{1}{l} \sum_{i=1}^l  x_i \\
2 \frac{1}{l} \sum_{i=1}^l  x_i & 2 \frac{1}{l} \sum_{i=1}^l  x_i^2
\end{pmatrix}
\]

    \begin{tcolorbox}[breakable, size=fbox, boxrule=1pt, pad at break*=1mm,colback=cellbackground, colframe=cellborder]
\prompt{In}{incolor}{9}{\boxspacing}
\begin{Verbatim}[commandchars=\\\{\}]
\PY{o}{\PYZpc{}\PYZpc{}time}
\PY{n}{history} \PY{o}{=} \PY{n}{np}\PY{o}{.}\PY{n}{array}\PY{p}{(}\PY{p}{[}\PY{p}{]}\PY{p}{)}
\PY{n}{w} \PY{o}{=} \PY{n}{np}\PY{o}{.}\PY{n}{zeros}\PY{p}{(}\PY{n}{shape}\PY{o}{=}\PY{n}{X}\PY{o}{.}\PY{n}{shape}\PY{p}{[}\PY{l+m+mi}{1}\PY{p}{]}\PY{p}{)}

\PY{n}{steps} \PY{o}{=} \PY{l+m+mi}{100}
\PY{k}{for} \PY{n}{\PYZus{}} \PY{o+ow}{in} \PY{n+nb}{range}\PY{p}{(}\PY{n}{steps}\PY{p}{)}\PY{p}{:}
     \PY{n}{grad} \PY{o}{=} \PY{n}{np}\PY{o}{.}\PY{n}{array}\PY{p}{(}\PY{p}{[}\PY{o}{\PYZhy{}}\PY{l+m+mi}{2} \PY{o}{*} \PY{p}{(}\PY{n}{y} \PY{o}{\PYZhy{}} \PY{n}{X} \PY{o}{@} \PY{n}{w}\PY{p}{[}\PY{p}{:}\PY{p}{:}\PY{o}{\PYZhy{}}\PY{l+m+mi}{1}\PY{p}{]}\PY{p}{)}\PY{o}{.}\PY{n}{mean}\PY{p}{(}\PY{p}{)}\PY{p}{,}
                      \PY{o}{\PYZhy{}}\PY{l+m+mi}{2} \PY{o}{*} \PY{p}{(}\PY{n}{X}\PY{p}{[}\PY{p}{:}\PY{p}{,}\PY{l+m+mi}{0}\PY{p}{]} \PY{o}{*} \PY{p}{(}\PY{n}{y} \PY{o}{\PYZhy{}} \PY{n}{X} \PY{o}{@} \PY{n}{w}\PY{p}{[}\PY{p}{:}\PY{p}{:}\PY{o}{\PYZhy{}}\PY{l+m+mi}{1}\PY{p}{]}\PY{p}{)}\PY{p}{)}\PY{o}{.}\PY{n}{mean}\PY{p}{(}\PY{p}{)}
                    \PY{p}{]}\PY{p}{)}
     \PY{n}{H} \PY{o}{=} \PY{n}{np}\PY{o}{.}\PY{n}{array}\PY{p}{(}
         \PY{p}{[}
             \PY{p}{[}\PY{l+m+mi}{2}\PY{p}{,} \PY{l+m+mi}{2}\PY{o}{*}\PY{n}{np}\PY{o}{.}\PY{n}{mean}\PY{p}{(}\PY{n}{X}\PY{p}{[}\PY{p}{:}\PY{p}{,}\PY{l+m+mi}{0}\PY{p}{]}\PY{p}{)}\PY{p}{]}\PY{p}{,}
             \PY{p}{[}\PY{l+m+mi}{2}\PY{o}{*}\PY{n}{np}\PY{o}{.}\PY{n}{mean}\PY{p}{(}\PY{n}{X}\PY{p}{[}\PY{p}{:}\PY{p}{,}\PY{l+m+mi}{0}\PY{p}{]}\PY{p}{)}\PY{p}{,} \PY{l+m+mi}{2}\PY{o}{*}\PY{n}{np}\PY{o}{.}\PY{n}{mean}\PY{p}{(}\PY{n}{X}\PY{p}{[}\PY{p}{:}\PY{p}{,}\PY{l+m+mi}{0}\PY{p}{]}\PY{o}{*}\PY{o}{*}\PY{l+m+mi}{2}\PY{p}{)}\PY{p}{]}
         \PY{p}{]}
     \PY{p}{)}
     \PY{n}{w} \PY{o}{=} \PY{n}{w} \PY{o}{\PYZhy{}} \PY{n}{np}\PY{o}{.}\PY{n}{linalg}\PY{o}{.}\PY{n}{inv}\PY{p}{(}\PY{n}{H}\PY{p}{)}\PY{n+nd}{@grad}
     \PY{n}{history} \PY{o}{=} \PY{n}{np}\PY{o}{.}\PY{n}{append}\PY{p}{(}\PY{n}{history}\PY{p}{,} \PY{n}{mse}\PY{p}{(}\PY{o}{*}\PY{n}{w}\PY{p}{)}\PY{p}{)}
\PY{n+nb}{print}\PY{p}{(}\PY{l+s+s1}{\PYZsq{}}\PY{l+s+s1}{полученное решение:}\PY{l+s+s1}{\PYZsq{}}\PY{p}{,}\PY{n}{w}\PY{p}{[}\PY{p}{:}\PY{p}{:}\PY{o}{\PYZhy{}}\PY{l+m+mi}{1}\PY{p}{]}\PY{p}{)}
\PY{n+nb}{print}\PY{p}{(}\PY{l+s+s1}{\PYZsq{}}\PY{l+s+s1}{MSE = }\PY{l+s+si}{\PYZob{}\PYZcb{}}\PY{l+s+s1}{\PYZsq{}}\PY{o}{.}\PY{n}{format}\PY{p}{(}\PY{n}{history}\PY{p}{[}\PY{o}{\PYZhy{}}\PY{l+m+mi}{1}\PY{p}{]}\PY{p}{)}\PY{p}{)}
\PY{n}{plt}\PY{o}{.}\PY{n}{title}\PY{p}{(}\PY{l+s+s1}{\PYZsq{}}\PY{l+s+s1}{MSE}\PY{l+s+s1}{\PYZsq{}}\PY{p}{)}
\PY{n}{plt}\PY{o}{.}\PY{n}{xlabel}\PY{p}{(}\PY{l+s+s1}{\PYZsq{}}\PY{l+s+s1}{номер итерации}\PY{l+s+s1}{\PYZsq{}}\PY{p}{)}
\PY{n}{plt}\PY{o}{.}\PY{n}{plot}\PY{p}{(}\PY{n}{history}\PY{p}{)}
\PY{n}{plt}\PY{o}{.}\PY{n}{show}\PY{p}{(}\PY{p}{)}
\end{Verbatim}
\end{tcolorbox}

    \begin{Verbatim}[commandchars=\\\{\}]
полученное решение: [43.08913515 20.58255766]
MSE = 19.513562605309744
    \end{Verbatim}

    \begin{center}
    \adjustimage{max size={0.9\linewidth}{0.9\paperheight}}{homework_files/homework_20_1.pdf}
    \end{center}
    { \hspace*{\fill} \\}
    
    \begin{Verbatim}[commandchars=\\\{\}]
CPU times: user 422 ms, sys: 26 ms, total: 448 ms
Wall time: 432 ms
    \end{Verbatim}

    \hypertarget{ux445ux43cux43cux43cux43cux43c.}{%
\subsubsection{Хммммм\ldots.}\label{ux445ux43cux43cux43cux43cux43c.}}

Судя по всему, алгоритм нашел лучшее решение уже после первой итерации,
причем оно в точности совпадает с полученным аналитически. Какой из
этого можно сделать вывод?\\
Метод Нютона относится к методам оптимизации второго порядка(в них
используется Гессиан). А для них доказано, что в случае квадратичной
функции, они находят экстремум за 1 итерацию, что, собственно, я и
получил. Мне следовало бы учесть этот факт заранее, но, увы, курса по
методам оптимизации у меня пока не было, поэтому знания я получал в
процессе работы.

В планах у меня был разбор квази-ньютоновских методов типа \textbf{BFGS}
(они называются квазиньютоновскими, потому что в них Гессиан не
вычисляется напрямую, а используется некоторая его аппроксимация), но
после полученного выше результата реализовывать их мне стало
неинтересно, поэтому рассмотрим методы нулевого и первого порядка.

    \begin{tcolorbox}[breakable, size=fbox, boxrule=1pt, pad at break*=1mm,colback=cellbackground, colframe=cellborder]
\prompt{In}{incolor}{10}{\boxspacing}
\begin{Verbatim}[commandchars=\\\{\}]
\PY{k+kn}{from} \PY{n+nn}{scipy}\PY{n+nn}{.}\PY{n+nn}{optimize} \PY{k+kn}{import} \PY{n}{minimize}


\PY{k}{def} \PY{n+nf}{super\PYZus{}mse}\PY{p}{(}\PY{n}{w}\PY{p}{)}\PY{p}{:}
    \PY{k}{return} \PY{p}{(}\PY{p}{(}\PY{n}{X}\PY{n+nd}{@w} \PY{o}{\PYZhy{}} \PY{n}{y}\PY{p}{)}\PY{o}{*}\PY{o}{*}\PY{l+m+mi}{2}\PY{p}{)}\PY{o}{.}\PY{n}{mean}\PY{p}{(}\PY{p}{)}

\PY{n}{np}\PY{o}{.}\PY{n}{random}\PY{o}{.}\PY{n}{seed}\PY{p}{(}\PY{l+m+mi}{42}\PY{p}{)}
\PY{n}{x0} \PY{o}{=} \PY{n}{np}\PY{o}{.}\PY{n}{random}\PY{o}{.}\PY{n}{randn}\PY{p}{(}\PY{l+m+mi}{2}\PY{p}{)}
\PY{n}{minimize}\PY{p}{(}\PY{n}{super\PYZus{}mse}\PY{p}{,} \PY{n}{x0}\PY{p}{,} \PY{n}{method}\PY{o}{=}\PY{l+s+s1}{\PYZsq{}}\PY{l+s+s1}{BFGS}\PY{l+s+s1}{\PYZsq{}}\PY{p}{)}
\end{Verbatim}
\end{tcolorbox}

            \begin{tcolorbox}[breakable, size=fbox, boxrule=.5pt, pad at break*=1mm, opacityfill=0]
\prompt{Out}{outcolor}{10}{\boxspacing}
\begin{Verbatim}[commandchars=\\\{\}]
      fun: 19.513562605309847
 hess\_inv: array([[0.61234345, 0.0635944 ],
       [0.0635944 , 0.50660863]])
      jac: array([ 9.53674316e-07, -2.38418579e-07])
  message: 'Optimization terminated successfully.'
     nfev: 40
      nit: 7
     njev: 10
   status: 0
  success: True
        x: array([43.08913541, 20.58255745])
\end{Verbatim}
\end{tcolorbox}
        
    \hypertarget{ux43cux435ux442ux43eux434-ux43dux435ux43bux434ux435ux440ux430-ux43cux438ux434ux430}{%
\subsection{Метод Нелдера ---
Мида}\label{ux43cux435ux442ux43eux434-ux43dux435ux43bux434ux435ux440ux430-ux43cux438ux434ux430}}

На мой взгляд, очень красивый метод безусловной (использует только
значения функции) оптимизации. Как работает?\\
Создается случаный симплекс(в нашем двумерном случае, это просто
треугольник). Считается значение функции в точках - вершинах
треугольника, дальше происходит череда отражений-растяжений-сжатий, в
результате которых мы получаем симплекс, который ближе к минимуму.

Можно выделить основыне особенности алгоритма:\\
• Функция не обязана быть гладкой\\
• Удобно использовать этот метод, когда вычислять значение функции
тяжело, так как он делает примерно 3 вычисления на каждой итерации\\
• Можно использовать для дискретной оптимизации\\
• Может сойтись к локальному минимуму :(\\
• Теоретическая сходимость метода не доказана, то есть никто не
гарантирует его сходимость

    \begin{tcolorbox}[breakable, size=fbox, boxrule=1pt, pad at break*=1mm,colback=cellbackground, colframe=cellborder]
\prompt{In}{incolor}{11}{\boxspacing}
\begin{Verbatim}[commandchars=\\\{\}]
\PY{n}{minimize}\PY{p}{(}\PY{n}{super\PYZus{}mse}\PY{p}{,} \PY{n}{x0}\PY{p}{,} \PY{n}{method}\PY{o}{=}\PY{l+s+s1}{\PYZsq{}}\PY{l+s+s1}{Nelder\PYZhy{}Mead}\PY{l+s+s1}{\PYZsq{}}\PY{p}{)}
\end{Verbatim}
\end{tcolorbox}

            \begin{tcolorbox}[breakable, size=fbox, boxrule=.5pt, pad at break*=1mm, opacityfill=0]
\prompt{Out}{outcolor}{11}{\boxspacing}
\begin{Verbatim}[commandchars=\\\{\}]
 final\_simplex: (array([[43.08916464, 20.58254066],
       [43.08909447, 20.58255763],
       [43.08910618, 20.58251746]]), array([19.51356261, 19.51356261,
19.51356261]))
           fun: 19.513562606422354
       message: 'Optimization terminated successfully.'
          nfev: 147
           nit: 79
        status: 0
       success: True
             x: array([43.08916464, 20.58254066])
\end{Verbatim}
\end{tcolorbox}
        
    \begin{tcolorbox}[breakable, size=fbox, boxrule=1pt, pad at break*=1mm,colback=cellbackground, colframe=cellborder]
\prompt{In}{incolor}{12}{\boxspacing}
\begin{Verbatim}[commandchars=\\\{\}]
\PY{o}{\PYZpc{}\PYZpc{}time}
\PY{n}{alpha} \PY{o}{=} \PY{l+m+mi}{1}
\PY{n}{beta} \PY{o}{=} \PY{l+m+mf}{0.5}
\PY{n}{gamma} \PY{o}{=} \PY{l+m+mi}{2}
\PY{n}{maxiter} \PY{o}{=} \PY{l+m+mi}{50}
\PY{n}{eps} \PY{o}{=} \PY{l+m+mf}{1e\PYZhy{}5}

\PY{c+c1}{\PYZsh{} подготовка}
\PY{n}{v} \PY{o}{=} \PY{n}{np}\PY{o}{.}\PY{n}{array}\PY{p}{(}\PY{p}{[}
    \PY{p}{[}\PY{l+m+mi}{1}\PY{p}{,}\PY{l+m+mi}{2}\PY{p}{]}\PY{p}{,}
    \PY{p}{[}\PY{l+m+mi}{4}\PY{p}{,}\PY{l+m+mi}{2}\PY{p}{]}\PY{p}{,}
    \PY{p}{[}\PY{l+m+mi}{5}\PY{p}{,}\PY{l+m+mf}{6.5}\PY{p}{]}
    \PY{p}{]}\PY{p}{)}
\PY{n}{history} \PY{o}{=} \PY{n}{np}\PY{o}{.}\PY{n}{array}\PY{p}{(}\PY{p}{[}\PY{p}{]}\PY{p}{)}

\PY{n}{i} \PY{o}{=} \PY{l+m+mi}{0}
\PY{c+c1}{\PYZsh{} for i in range(maxiter):}
\PY{k}{while} \PY{n}{i} \PY{o}{\PYZlt{}} \PY{n}{maxiter} \PY{o+ow}{and} \PY{n}{np}\PY{o}{.}\PY{n}{var}\PY{p}{(}\PY{n}{v}\PY{p}{,} \PY{n}{axis}\PY{o}{=}\PY{l+m+mi}{0}\PY{p}{)}\PY{o}{.}\PY{n}{mean}\PY{p}{(}\PY{p}{)} \PY{o}{\PYZgt{}} \PY{n}{eps} \PY{p}{:}
    \PY{n}{i} \PY{o}{+}\PY{o}{=} \PY{l+m+mi}{1}
    \PY{c+c1}{\PYZsh{} сортируем точки симплекса по убыванию функции}
    \PY{n}{values} \PY{o}{=} \PY{n}{np}\PY{o}{.}\PY{n}{vectorize}\PY{p}{(}\PY{n}{mse}\PY{p}{)}\PY{p}{(}\PY{n}{v}\PY{p}{[}\PY{p}{:}\PY{p}{,}\PY{l+m+mi}{0}\PY{p}{]}\PY{p}{,}\PY{n}{v}\PY{p}{[}\PY{p}{:}\PY{p}{,}\PY{l+m+mi}{1}\PY{p}{]}\PY{p}{)}
    \PY{n}{ind} \PY{o}{=} \PY{n}{np}\PY{o}{.}\PY{n}{argsort}\PY{p}{(}\PY{n}{values}\PY{p}{)}\PY{o}{.}\PY{n}{reshape}\PY{p}{(}\PY{o}{\PYZhy{}}\PY{l+m+mi}{1}\PY{p}{,}\PY{l+m+mi}{1}\PY{p}{)}\PY{p}{[}\PY{p}{:}\PY{p}{:}\PY{o}{\PYZhy{}}\PY{l+m+mi}{1}\PY{p}{]}
    \PY{n}{v} \PY{o}{=} \PY{n}{np}\PY{o}{.}\PY{n}{take\PYZus{}along\PYZus{}axis}\PY{p}{(}\PY{n}{v}\PY{p}{,} \PY{n}{ind}\PY{p}{,} \PY{n}{axis}\PY{o}{=}\PY{l+m+mi}{0}\PY{p}{)}
    \PY{n}{history} \PY{o}{=} \PY{n}{np}\PY{o}{.}\PY{n}{append}\PY{p}{(}\PY{n}{history}\PY{p}{,} \PY{n}{mse}\PY{p}{(}\PY{o}{*}\PY{n}{v}\PY{p}{[}\PY{o}{\PYZhy{}}\PY{l+m+mi}{1}\PY{p}{]}\PY{p}{)}\PY{p}{)}

    \PY{n}{fh} \PY{o}{=} \PY{n}{mse}\PY{p}{(}\PY{o}{*}\PY{n}{v}\PY{p}{[}\PY{l+m+mi}{0}\PY{p}{]}\PY{p}{)}
    \PY{n}{fg} \PY{o}{=} \PY{n}{mse}\PY{p}{(}\PY{o}{*}\PY{n}{v}\PY{p}{[}\PY{l+m+mi}{1}\PY{p}{]}\PY{p}{)}
    \PY{n}{fl} \PY{o}{=} \PY{n}{mse}\PY{p}{(}\PY{o}{*}\PY{n}{v}\PY{p}{[}\PY{l+m+mi}{2}\PY{p}{]}\PY{p}{)}
    \PY{c+c1}{\PYZsh{} считаем центр масс симплекса}
    \PY{n}{xc} \PY{o}{=} \PY{n}{v}\PY{p}{[}\PY{l+m+mi}{1}\PY{p}{:}\PY{p}{]}\PY{o}{.}\PY{n}{mean}\PY{p}{(}\PY{n}{axis}\PY{o}{=}\PY{l+m+mi}{0}\PY{p}{)}

    \PY{c+c1}{\PYZsh{} отражаем худшую точку относительно симплекса}
    \PY{n}{xr} \PY{o}{=} \PY{p}{(}\PY{l+m+mi}{1} \PY{o}{+} \PY{n}{alpha}\PY{p}{)} \PY{o}{*} \PY{n}{xc} \PY{o}{\PYZhy{}} \PY{n}{alpha} \PY{o}{*} \PY{n}{v}\PY{p}{[}\PY{l+m+mi}{0}\PY{p}{]}
    \PY{n}{fr} \PY{o}{=} \PY{n}{mse}\PY{p}{(}\PY{o}{*}\PY{n}{xr}\PY{p}{)}

    \PY{c+c1}{\PYZsh{} если в отраженной точке значение меньше чем в лучшей}
    \PY{k}{if} \PY{n}{fr} \PY{o}{\PYZlt{}} \PY{n}{fl}\PY{p}{:}
        \PY{c+c1}{\PYZsh{} пробуем растянуть}
        \PY{n}{xe} \PY{o}{=} \PY{p}{(}\PY{l+m+mi}{1} \PY{o}{\PYZhy{}} \PY{n}{gamma}\PY{p}{)} \PY{o}{*} \PY{n}{xc} \PY{o}{+} \PY{n}{gamma} \PY{o}{*} \PY{n}{xr}
        \PY{n}{fe} \PY{o}{=} \PY{n}{mse}\PY{p}{(}\PY{o}{*}\PY{n}{xe}\PY{p}{)}
        \PY{k}{if} \PY{n}{fe} \PY{o}{\PYZlt{}} \PY{n}{fr}\PY{p}{:}
            \PY{c+c1}{\PYZsh{}расширим симплекс}
            \PY{n}{v}\PY{p}{[}\PY{l+m+mi}{0}\PY{p}{]} \PY{o}{=} \PY{n}{xe}
            \PY{k}{continue}
        \PY{k}{if} \PY{n}{fr} \PY{o}{\PYZlt{}} \PY{n}{fe}\PY{p}{:}
            \PY{c+c1}{\PYZsh{} оставим просто отраженный}
            \PY{n}{v}\PY{p}{[}\PY{l+m+mi}{0}\PY{p}{]} \PY{o}{=} \PY{n}{xr}
            \PY{k}{continue}
    \PY{k}{if} \PY{n}{fl} \PY{o}{\PYZlt{}} \PY{n}{fr} \PY{o+ow}{and} \PY{n}{fr} \PY{o}{\PYZlt{}} \PY{n}{fg}\PY{p}{:}
        \PY{n}{v}\PY{p}{[}\PY{l+m+mi}{0}\PY{p}{]} \PY{o}{=} \PY{n}{xr}
        \PY{k}{continue}
    \PY{k}{if} \PY{n}{fg} \PY{o}{\PYZlt{}} \PY{n}{fr} \PY{o+ow}{and} \PY{n}{fr} \PY{o}{\PYZlt{}} \PY{n}{fh}\PY{p}{:}
        \PY{n}{xr}\PY{p}{,} \PY{n}{v}\PY{p}{[}\PY{l+m+mi}{0}\PY{p}{]} \PY{o}{=} \PY{n}{v}\PY{p}{[}\PY{l+m+mi}{0}\PY{p}{]}\PY{p}{,} \PY{n}{xr}
        \PY{n}{fh}\PY{p}{,} \PY{n}{fr} \PY{o}{=} \PY{n}{fr}\PY{p}{,} \PY{n}{fh}

    \PY{c+c1}{\PYZsh{} сжатие}
    \PY{n}{xs} \PY{o}{=} \PY{n}{beta} \PY{o}{*} \PY{n}{v}\PY{p}{[}\PY{l+m+mi}{0}\PY{p}{]} \PY{o}{+} \PY{p}{(}\PY{l+m+mi}{1} \PY{o}{\PYZhy{}} \PY{n}{beta}\PY{p}{)} \PY{o}{*} \PY{n}{xc}
    \PY{n}{fs} \PY{o}{=} \PY{n}{mse}\PY{p}{(}\PY{o}{*}\PY{n}{xs}\PY{p}{)}

    \PY{k}{if} \PY{n}{fs} \PY{o}{\PYZlt{}} \PY{n}{fh}\PY{p}{:}
        \PY{n}{v}\PY{p}{[}\PY{l+m+mi}{0}\PY{p}{]} \PY{o}{=} \PY{n}{xs}
        \PY{k}{continue}
    \PY{k}{if} \PY{n}{fs} \PY{o}{\PYZgt{}} \PY{n}{fh}\PY{p}{:}
        \PY{k}{for} \PY{n}{j} \PY{o+ow}{in} \PY{n+nb}{range}\PY{p}{(}\PY{l+m+mi}{2}\PY{p}{)}\PY{p}{:}
            \PY{n}{v}\PY{p}{[}\PY{n}{j}\PY{p}{]} \PY{o}{=} \PY{n}{v}\PY{p}{[}\PY{l+m+mi}{2}\PY{p}{]} \PY{o}{+} \PY{p}{(}\PY{n}{v}\PY{p}{[}\PY{n}{j}\PY{p}{]} \PY{o}{\PYZhy{}} \PY{n}{v}\PY{p}{[}\PY{l+m+mi}{2}\PY{p}{]}\PY{p}{)}\PY{o}{/}\PY{l+m+mi}{2}


\PY{n}{w} \PY{o}{=} \PY{n}{v}\PY{p}{[}\PY{o}{\PYZhy{}}\PY{l+m+mi}{1}\PY{p}{]}
\PY{n+nb}{print}\PY{p}{(}\PY{l+s+s1}{\PYZsq{}}\PY{l+s+s1}{полученное решение:}\PY{l+s+s1}{\PYZsq{}}\PY{p}{,}\PY{n}{w}\PY{p}{[}\PY{p}{:}\PY{p}{:}\PY{o}{\PYZhy{}}\PY{l+m+mi}{1}\PY{p}{]}\PY{p}{)}
\PY{n+nb}{print}\PY{p}{(}\PY{l+s+s1}{\PYZsq{}}\PY{l+s+s1}{MSE = }\PY{l+s+si}{\PYZob{}\PYZcb{}}\PY{l+s+s1}{\PYZsq{}}\PY{o}{.}\PY{n}{format}\PY{p}{(}\PY{n}{history}\PY{p}{[}\PY{o}{\PYZhy{}}\PY{l+m+mi}{1}\PY{p}{]}\PY{p}{)}\PY{p}{)}
\PY{n}{plt}\PY{o}{.}\PY{n}{title}\PY{p}{(}\PY{l+s+s1}{\PYZsq{}}\PY{l+s+s1}{MSE}\PY{l+s+s1}{\PYZsq{}}\PY{p}{)}
\PY{n}{plt}\PY{o}{.}\PY{n}{xlabel}\PY{p}{(}\PY{l+s+s1}{\PYZsq{}}\PY{l+s+s1}{номер итерации}\PY{l+s+s1}{\PYZsq{}}\PY{p}{)}
\PY{n}{plt}\PY{o}{.}\PY{n}{plot}\PY{p}{(}\PY{n}{history}\PY{p}{)}
\PY{n}{plt}\PY{o}{.}\PY{n}{show}\PY{p}{(}\PY{p}{)}
\end{Verbatim}
\end{tcolorbox}

    \begin{Verbatim}[commandchars=\\\{\}]
полученное решение: [43.08526461 20.58486516]
MSE = 19.513582178760267
    \end{Verbatim}

    \begin{center}
    \adjustimage{max size={0.9\linewidth}{0.9\paperheight}}{homework_files/homework_25_1.pdf}
    \end{center}
    { \hspace*{\fill} \\}
    
    \begin{Verbatim}[commandchars=\\\{\}]
CPU times: user 309 ms, sys: 11.4 ms, total: 320 ms
Wall time: 365 ms
    \end{Verbatim}

    \textbf{УРАААА!!!!}\\
Алгоритм сошелся к уже известному нам решению (отладка заняла кучу
времени, так что на работу я потратил точно больше 10 часов :-))\\
Причем для хорошего решения хватает уже 20 итераций

    \hypertarget{ux430ux43bux433ux43eux440ux438ux442ux43c-ux438ux43cux438ux442ux430ux446ux438ux438-ux43eux442ux436ux438ux433ux430}{%
\subsection{Алгоритм имитации
отжига}\label{ux430ux43bux433ux43eux440ux438ux442ux43c-ux438ux43cux438ux442ux430ux446ux438ux438-ux43eux442ux436ux438ux433ux430}}

Алгоритм подсмотрен у природы, а именно авторы вдохновлялись процессом
кристаллизации вещества, когда переходы атомов из одного состояния в
другое происходят с некоторой вероятностью, которая хитро считается и
изменяется во времени. Посмотреть подробнее можно в
\href{https://ru.wikipedia.org/wiki/\%D0\%90\%D0\%BB\%D0\%B3\%D0\%BE\%D1\%80\%D0\%B8\%D1\%82\%D0\%BC_\%D0\%B8\%D0\%BC\%D0\%B8\%D1\%82\%D0\%B0\%D1\%86\%D0\%B8\%D0\%B8_\%D0\%BE\%D1\%82\%D0\%B6\%D0\%B8\%D0\%B3\%D0\%B0}{вики}

    \begin{tcolorbox}[breakable, size=fbox, boxrule=1pt, pad at break*=1mm,colback=cellbackground, colframe=cellborder]
\prompt{In}{incolor}{13}{\boxspacing}
\begin{Verbatim}[commandchars=\\\{\}]
\PY{c+c1}{\PYZsh{} генерировать случайное приближение }
\PY{n}{r} \PY{o}{=} \PY{l+m+mi}{100}
\PY{k}{def} \PY{n+nf}{generate\PYZus{}new}\PY{p}{(}\PY{n}{cur\PYZus{}w}\PY{p}{,} \PY{n}{cur\PYZus{}T}\PY{p}{)}\PY{p}{:}
    \PY{k}{return} \PY{n}{np}\PY{o}{.}\PY{n}{random}\PY{o}{.}\PY{n}{random\PYZus{}sample}\PY{p}{(}\PY{l+m+mi}{2}\PY{p}{)} \PY{o}{*} \PY{p}{(}\PY{l+m+mi}{2} \PY{o}{*} \PY{n}{r}\PY{p}{)} \PY{o}{*} \PY{n}{cur\PYZus{}T} \PY{o}{+} \PY{n}{cur\PYZus{}w} \PY{o}{\PYZhy{}} \PY{n}{r} \PY{o}{*} \PY{n}{cur\PYZus{}T} 

\PY{c+c1}{\PYZsh{} распределение гиббса}
\PY{k}{def} \PY{n+nf}{gibbs\PYZus{}pdf}\PY{p}{(}\PY{n}{x}\PY{p}{)}\PY{p}{:}
    \PY{k}{return} \PY{n}{np}\PY{o}{.}\PY{n}{exp}\PY{p}{(}\PY{o}{\PYZhy{}}\PY{n}{x}\PY{p}{)}
\end{Verbatim}
\end{tcolorbox}

    \hypertarget{ux43cux43eux44f-ux440ux435ux430ux43bux438ux437ux430ux446ux438ux44f}{%
\subsubsection{Моя
реализация}\label{ux43cux43eux44f-ux440ux435ux430ux43bux438ux437ux430ux446ux438ux44f}}

Температура убывает по закону: \[
T = \alpha T, \alpha \in (0,1)
\]\\
Новая точка выбирается \textbf{ИЗОБРЕТЕННОЙ МНОЮ} эвристикой.\\
А именно, введем радиус поиска \(r\), затем будем выбирать новую точку
\(w^*\) из равномерного распределения
\(w^* \sim \mathcal{U}(w-rT,w+rT)\) , что можно интерпретировать так:
выбирается случайная точка вокруг текущего приближения из диапозона,
который убывает вместе с температурой.\\
Затем считается \(MSE(w*)\). Если новое приближение оказалось лучше
предыдущего \(MSE(w^*)<MSE(w)\), то принимаем новое приближение и
переходим к следующей итерации. Иначе решение о переходе принимается по
правилу: \[
\exp \left( -\frac{MSE(w^*) - MSE(w)}{T} \right)  > 0.5
\]

    \begin{tcolorbox}[breakable, size=fbox, boxrule=1pt, pad at break*=1mm,colback=cellbackground, colframe=cellborder]
\prompt{In}{incolor}{14}{\boxspacing}
\begin{Verbatim}[commandchars=\\\{\}]
\PY{o}{\PYZpc{}\PYZpc{}time}
\PY{c+c1}{\PYZsh{} начальная температура}
\PY{n}{T} \PY{o}{=} \PY{l+m+mi}{100}
\PY{c+c1}{\PYZsh{} насколько будет убывать температура}
\PY{n}{alpha} \PY{o}{=} \PY{l+m+mf}{0.9}
\PY{c+c1}{\PYZsh{} инициализация}
\PY{n}{w} \PY{o}{=} \PY{n}{np}\PY{o}{.}\PY{n}{random}\PY{o}{.}\PY{n}{random\PYZus{}sample}\PY{p}{(}\PY{l+m+mi}{2}\PY{p}{)}
\PY{n}{history} \PY{o}{=} \PY{n}{np}\PY{o}{.}\PY{n}{array}\PY{p}{(}\PY{p}{[}\PY{p}{]}\PY{p}{)}

\PY{n}{i\PYZus{}max} \PY{o}{=} \PY{l+m+mi}{100}
\PY{n}{i} \PY{o}{=} \PY{l+m+mi}{0}
\PY{k}{while} \PY{n}{i} \PY{o}{\PYZlt{}} \PY{n}{i\PYZus{}max}\PY{p}{:}
    \PY{n}{history} \PY{o}{=} \PY{n}{np}\PY{o}{.}\PY{n}{append}\PY{p}{(}\PY{n}{history}\PY{p}{,} \PY{n}{mse}\PY{p}{(}\PY{o}{*}\PY{n}{w}\PY{p}{)}\PY{p}{)}
    \PY{n}{T} \PY{o}{=} \PY{n}{T} \PY{o}{*} \PY{n}{alpha}
    \PY{n}{w\PYZus{}new} \PY{o}{=} \PY{n}{generate\PYZus{}new}\PY{p}{(}\PY{n}{w}\PY{p}{,} \PY{n}{T}\PY{p}{)}
    \PY{n}{f\PYZus{}cur} \PY{o}{=} \PY{n}{mse}\PY{p}{(}\PY{o}{*}\PY{n}{w}\PY{p}{)}
    \PY{n}{f\PYZus{}new} \PY{o}{=} \PY{n}{mse}\PY{p}{(}\PY{o}{*}\PY{n}{w\PYZus{}new}\PY{p}{)}
    \PY{n}{delta\PYZus{}f} \PY{o}{=} \PY{n}{f\PYZus{}new} \PY{o}{\PYZhy{}} \PY{n}{f\PYZus{}cur}
    \PY{k}{if} \PY{n}{delta\PYZus{}f} \PY{o}{\PYZlt{}} \PY{l+m+mi}{0}\PY{p}{:}
        \PY{n}{w} \PY{o}{=} \PY{n}{w\PYZus{}new}
    \PY{k}{else}\PY{p}{:}
        \PY{k}{if} \PY{n}{gibbs\PYZus{}pdf}\PY{p}{(}\PY{n}{delta\PYZus{}f} \PY{o}{/} \PY{n}{T}\PY{p}{)} \PY{o}{\PYZgt{}} \PY{l+m+mf}{0.5}\PY{p}{:}
            \PY{n}{w} \PY{o}{=} \PY{n}{w\PYZus{}new}
    \PY{n}{i} \PY{o}{+}\PY{o}{=} \PY{l+m+mi}{1}


\PY{n+nb}{print}\PY{p}{(}\PY{l+s+s1}{\PYZsq{}}\PY{l+s+s1}{полученное решение:}\PY{l+s+s1}{\PYZsq{}}\PY{p}{,}\PY{n}{w}\PY{p}{[}\PY{p}{:}\PY{p}{:}\PY{o}{\PYZhy{}}\PY{l+m+mi}{1}\PY{p}{]}\PY{p}{)}
\PY{n+nb}{print}\PY{p}{(}\PY{l+s+s1}{\PYZsq{}}\PY{l+s+s1}{MSE = }\PY{l+s+si}{\PYZob{}\PYZcb{}}\PY{l+s+s1}{\PYZsq{}}\PY{o}{.}\PY{n}{format}\PY{p}{(}\PY{n}{history}\PY{p}{[}\PY{o}{\PYZhy{}}\PY{l+m+mi}{1}\PY{p}{]}\PY{p}{)}\PY{p}{)}
\PY{n}{plt}\PY{o}{.}\PY{n}{title}\PY{p}{(}\PY{l+s+s1}{\PYZsq{}}\PY{l+s+s1}{MSE}\PY{l+s+s1}{\PYZsq{}}\PY{p}{)}
\PY{n}{plt}\PY{o}{.}\PY{n}{xlabel}\PY{p}{(}\PY{l+s+s1}{\PYZsq{}}\PY{l+s+s1}{номер итерации}\PY{l+s+s1}{\PYZsq{}}\PY{p}{)}
\PY{n}{plt}\PY{o}{.}\PY{n}{plot}\PY{p}{(}\PY{n}{history}\PY{p}{)}
\PY{n}{plt}\PY{o}{.}\PY{n}{show}\PY{p}{(}\PY{p}{)}
\end{Verbatim}
\end{tcolorbox}

    \begin{Verbatim}[commandchars=\\\{\}]
полученное решение: [43.0658111  20.67378853]
MSE = 19.522777685764837
    \end{Verbatim}

    \begin{center}
    \adjustimage{max size={0.9\linewidth}{0.9\paperheight}}{homework_files/homework_30_1.pdf}
    \end{center}
    { \hspace*{\fill} \\}
    
    \begin{Verbatim}[commandchars=\\\{\}]
CPU times: user 311 ms, sys: 12.7 ms, total: 324 ms
Wall time: 337 ms
    \end{Verbatim}

    УРА! Алгоритм сошелся к минимуму! Теперь обсудим особенности метода:\\
• Не требует гладкости функции\\
• Может использоваться дл дискретной, комбинаторной оптимизации\\
• Может сам выпрыгнуть из локальных минимумов благодаря случайности
невыгодных переходов\\
• Может не сойтись, тогда можно попробовать выбрать другое начальное
приближение

    \hypertarget{ux438ux442ux43eux433ux438}{%
\subsection{Итоги}\label{ux438ux442ux43eux433ux438}}

Мы рассмотрели некоторые методы оптимизации на примере задачи нахождения
коэффициентов линейной регрессии, это было достаточно увлекательно. Все
методы успешно сработали и решили задачу. В дальнейшем можно провести
рефакторинг кода, оптимизировать вычисления \texttt{NumPy}, подумать над
критериями останова, потому что во многих методах я просто фиксировал
количество итераций.


    % Add a bibliography block to the postdoc
    
    
    
\end{document}
